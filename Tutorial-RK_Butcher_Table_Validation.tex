\documentclass[landscape,letterpaper,10pt,english]{article}

    \usepackage[breakable]{tcolorbox}
    \usepackage{parskip} % Stop auto-indenting (to mimic markdown behaviour)
    

    % Basic figure setup, for now with no caption control since it's done
    % automatically by Pandoc (which extracts ![](path) syntax from Markdown).
    \usepackage{graphicx}
    % Maintain compatibility with old templates. Remove in nbconvert 6.0
    \let\Oldincludegraphics\includegraphics
    % Ensure that by default, figures have no caption (until we provide a
    % proper Figure object with a Caption API and a way to capture that
    % in the conversion process - todo).
    \usepackage{caption}
    \DeclareCaptionFormat{nocaption}{}
    \captionsetup{format=nocaption,aboveskip=0pt,belowskip=0pt}

    \usepackage{float}
    \floatplacement{figure}{H} % forces figures to be placed at the correct location
    \usepackage{xcolor} % Allow colors to be defined
    \usepackage{enumerate} % Needed for markdown enumerations to work
    \usepackage{geometry} % Used to adjust the document margins
    \usepackage{amsmath} % Equations
    \usepackage{amssymb} % Equations
    \usepackage{textcomp} % defines textquotesingle
    % Hack from http://tex.stackexchange.com/a/47451/13684:
    \AtBeginDocument{%
        \def\PYZsq{\textquotesingle}% Upright quotes in Pygmentized code
    }
    \usepackage{upquote} % Upright quotes for verbatim code
    \usepackage{eurosym} % defines \euro

    \usepackage{iftex}
    \ifPDFTeX
        \usepackage[T1]{fontenc}
        \IfFileExists{alphabeta.sty}{
              \usepackage{alphabeta}
          }{
              \usepackage[mathletters]{ucs}
              \usepackage[utf8x]{inputenc}
          }
    \else
        \usepackage{fontspec}
        \usepackage{unicode-math}
    \fi

    \usepackage{fancyvrb} % verbatim replacement that allows latex
    \usepackage{grffile} % extends the file name processing of package graphics
                         % to support a larger range
    \makeatletter % fix for old versions of grffile with XeLaTeX
    \@ifpackagelater{grffile}{2019/11/01}
    {
      % Do nothing on new versions
    }
    {
      \def\Gread@@xetex#1{%
        \IfFileExists{"\Gin@base".bb}%
        {\Gread@eps{\Gin@base.bb}}%
        {\Gread@@xetex@aux#1}%
      }
    }
    \makeatother
    \usepackage[Export]{adjustbox} % Used to constrain images to a maximum size
    \adjustboxset{max size={0.9\linewidth}{0.9\paperheight}}

    % The hyperref package gives us a pdf with properly built
    % internal navigation ('pdf bookmarks' for the table of contents,
    % internal cross-reference links, web links for URLs, etc.)
    \usepackage{hyperref}
    % The default LaTeX title has an obnoxious amount of whitespace. By default,
    % titling removes some of it. It also provides customization options.
    \usepackage{titling}
    \usepackage{longtable} % longtable support required by pandoc >1.10
    \usepackage{booktabs}  % table support for pandoc > 1.12.2
    \usepackage{array}     % table support for pandoc >= 2.11.3
    \usepackage{calc}      % table minipage width calculation for pandoc >= 2.11.1
    \usepackage[inline]{enumitem} % IRkernel/repr support (it uses the enumerate* environment)
    \usepackage[normalem]{ulem} % ulem is needed to support strikethroughs (\sout)
                                % normalem makes italics be italics, not underlines
    \usepackage{mathrsfs}
    

    
    % Colors for the hyperref package
    \definecolor{urlcolor}{rgb}{0,.145,.698}
    \definecolor{linkcolor}{rgb}{.71,0.21,0.01}
    \definecolor{citecolor}{rgb}{.12,.54,.11}

    % ANSI colors
    \definecolor{ansi-black}{HTML}{3E424D}
    \definecolor{ansi-black-intense}{HTML}{282C36}
    \definecolor{ansi-red}{HTML}{E75C58}
    \definecolor{ansi-red-intense}{HTML}{B22B31}
    \definecolor{ansi-green}{HTML}{00A250}
    \definecolor{ansi-green-intense}{HTML}{007427}
    \definecolor{ansi-yellow}{HTML}{DDB62B}
    \definecolor{ansi-yellow-intense}{HTML}{B27D12}
    \definecolor{ansi-blue}{HTML}{208FFB}
    \definecolor{ansi-blue-intense}{HTML}{0065CA}
    \definecolor{ansi-magenta}{HTML}{D160C4}
    \definecolor{ansi-magenta-intense}{HTML}{A03196}
    \definecolor{ansi-cyan}{HTML}{60C6C8}
    \definecolor{ansi-cyan-intense}{HTML}{258F8F}
    \definecolor{ansi-white}{HTML}{C5C1B4}
    \definecolor{ansi-white-intense}{HTML}{A1A6B2}
    \definecolor{ansi-default-inverse-fg}{HTML}{FFFFFF}
    \definecolor{ansi-default-inverse-bg}{HTML}{000000}

    % common color for the border for error outputs.
    \definecolor{outerrorbackground}{HTML}{FFDFDF}

    % commands and environments needed by pandoc snippets
    % extracted from the output of `pandoc -s`
    \providecommand{\tightlist}{%
      \setlength{\itemsep}{0pt}\setlength{\parskip}{0pt}}
    \DefineVerbatimEnvironment{Highlighting}{Verbatim}{commandchars=\\\{\}}
    % Add ',fontsize=\small' for more characters per line
    \newenvironment{Shaded}{}{}
    \newcommand{\KeywordTok}[1]{\textcolor[rgb]{0.00,0.44,0.13}{\textbf{{#1}}}}
    \newcommand{\DataTypeTok}[1]{\textcolor[rgb]{0.56,0.13,0.00}{{#1}}}
    \newcommand{\DecValTok}[1]{\textcolor[rgb]{0.25,0.63,0.44}{{#1}}}
    \newcommand{\BaseNTok}[1]{\textcolor[rgb]{0.25,0.63,0.44}{{#1}}}
    \newcommand{\FloatTok}[1]{\textcolor[rgb]{0.25,0.63,0.44}{{#1}}}
    \newcommand{\CharTok}[1]{\textcolor[rgb]{0.25,0.44,0.63}{{#1}}}
    \newcommand{\StringTok}[1]{\textcolor[rgb]{0.25,0.44,0.63}{{#1}}}
    \newcommand{\CommentTok}[1]{\textcolor[rgb]{0.38,0.63,0.69}{\textit{{#1}}}}
    \newcommand{\OtherTok}[1]{\textcolor[rgb]{0.00,0.44,0.13}{{#1}}}
    \newcommand{\AlertTok}[1]{\textcolor[rgb]{1.00,0.00,0.00}{\textbf{{#1}}}}
    \newcommand{\FunctionTok}[1]{\textcolor[rgb]{0.02,0.16,0.49}{{#1}}}
    \newcommand{\RegionMarkerTok}[1]{{#1}}
    \newcommand{\ErrorTok}[1]{\textcolor[rgb]{1.00,0.00,0.00}{\textbf{{#1}}}}
    \newcommand{\NormalTok}[1]{{#1}}

    % Additional commands for more recent versions of Pandoc
    \newcommand{\ConstantTok}[1]{\textcolor[rgb]{0.53,0.00,0.00}{{#1}}}
    \newcommand{\SpecialCharTok}[1]{\textcolor[rgb]{0.25,0.44,0.63}{{#1}}}
    \newcommand{\VerbatimStringTok}[1]{\textcolor[rgb]{0.25,0.44,0.63}{{#1}}}
    \newcommand{\SpecialStringTok}[1]{\textcolor[rgb]{0.73,0.40,0.53}{{#1}}}
    \newcommand{\ImportTok}[1]{{#1}}
    \newcommand{\DocumentationTok}[1]{\textcolor[rgb]{0.73,0.13,0.13}{\textit{{#1}}}}
    \newcommand{\AnnotationTok}[1]{\textcolor[rgb]{0.38,0.63,0.69}{\textbf{\textit{{#1}}}}}
    \newcommand{\CommentVarTok}[1]{\textcolor[rgb]{0.38,0.63,0.69}{\textbf{\textit{{#1}}}}}
    \newcommand{\VariableTok}[1]{\textcolor[rgb]{0.10,0.09,0.49}{{#1}}}
    \newcommand{\ControlFlowTok}[1]{\textcolor[rgb]{0.00,0.44,0.13}{\textbf{{#1}}}}
    \newcommand{\OperatorTok}[1]{\textcolor[rgb]{0.40,0.40,0.40}{{#1}}}
    \newcommand{\BuiltInTok}[1]{{#1}}
    \newcommand{\ExtensionTok}[1]{{#1}}
    \newcommand{\PreprocessorTok}[1]{\textcolor[rgb]{0.74,0.48,0.00}{{#1}}}
    \newcommand{\AttributeTok}[1]{\textcolor[rgb]{0.49,0.56,0.16}{{#1}}}
    \newcommand{\InformationTok}[1]{\textcolor[rgb]{0.38,0.63,0.69}{\textbf{\textit{{#1}}}}}
    \newcommand{\WarningTok}[1]{\textcolor[rgb]{0.38,0.63,0.69}{\textbf{\textit{{#1}}}}}


    % Define a nice break command that doesn't care if a line doesn't already
    % exist.
    \def\br{\hspace*{\fill} \\* }
    % Math Jax compatibility definitions
    \def\gt{>}
    \def\lt{<}
    \let\Oldtex\TeX
    \let\Oldlatex\LaTeX
    \renewcommand{\TeX}{\textrm{\Oldtex}}
    \renewcommand{\LaTeX}{\textrm{\Oldlatex}}
    % Document parameters
    % Document title
    \title{Tutorial-RK\_Butcher\_Table\_Validation}
    
    
    
    
    
% Pygments definitions
\makeatletter
\def\PY@reset{\let\PY@it=\relax \let\PY@bf=\relax%
    \let\PY@ul=\relax \let\PY@tc=\relax%
    \let\PY@bc=\relax \let\PY@ff=\relax}
\def\PY@tok#1{\csname PY@tok@#1\endcsname}
\def\PY@toks#1+{\ifx\relax#1\empty\else%
    \PY@tok{#1}\expandafter\PY@toks\fi}
\def\PY@do#1{\PY@bc{\PY@tc{\PY@ul{%
    \PY@it{\PY@bf{\PY@ff{#1}}}}}}}
\def\PY#1#2{\PY@reset\PY@toks#1+\relax+\PY@do{#2}}

\@namedef{PY@tok@w}{\def\PY@tc##1{\textcolor[rgb]{0.73,0.73,0.73}{##1}}}
\@namedef{PY@tok@c}{\let\PY@it=\textit\def\PY@tc##1{\textcolor[rgb]{0.24,0.48,0.48}{##1}}}
\@namedef{PY@tok@cp}{\def\PY@tc##1{\textcolor[rgb]{0.61,0.40,0.00}{##1}}}
\@namedef{PY@tok@k}{\let\PY@bf=\textbf\def\PY@tc##1{\textcolor[rgb]{0.00,0.50,0.00}{##1}}}
\@namedef{PY@tok@kp}{\def\PY@tc##1{\textcolor[rgb]{0.00,0.50,0.00}{##1}}}
\@namedef{PY@tok@kt}{\def\PY@tc##1{\textcolor[rgb]{0.69,0.00,0.25}{##1}}}
\@namedef{PY@tok@o}{\def\PY@tc##1{\textcolor[rgb]{0.40,0.40,0.40}{##1}}}
\@namedef{PY@tok@ow}{\let\PY@bf=\textbf\def\PY@tc##1{\textcolor[rgb]{0.67,0.13,1.00}{##1}}}
\@namedef{PY@tok@nb}{\def\PY@tc##1{\textcolor[rgb]{0.00,0.50,0.00}{##1}}}
\@namedef{PY@tok@nf}{\def\PY@tc##1{\textcolor[rgb]{0.00,0.00,1.00}{##1}}}
\@namedef{PY@tok@nc}{\let\PY@bf=\textbf\def\PY@tc##1{\textcolor[rgb]{0.00,0.00,1.00}{##1}}}
\@namedef{PY@tok@nn}{\let\PY@bf=\textbf\def\PY@tc##1{\textcolor[rgb]{0.00,0.00,1.00}{##1}}}
\@namedef{PY@tok@ne}{\let\PY@bf=\textbf\def\PY@tc##1{\textcolor[rgb]{0.80,0.25,0.22}{##1}}}
\@namedef{PY@tok@nv}{\def\PY@tc##1{\textcolor[rgb]{0.10,0.09,0.49}{##1}}}
\@namedef{PY@tok@no}{\def\PY@tc##1{\textcolor[rgb]{0.53,0.00,0.00}{##1}}}
\@namedef{PY@tok@nl}{\def\PY@tc##1{\textcolor[rgb]{0.46,0.46,0.00}{##1}}}
\@namedef{PY@tok@ni}{\let\PY@bf=\textbf\def\PY@tc##1{\textcolor[rgb]{0.44,0.44,0.44}{##1}}}
\@namedef{PY@tok@na}{\def\PY@tc##1{\textcolor[rgb]{0.41,0.47,0.13}{##1}}}
\@namedef{PY@tok@nt}{\let\PY@bf=\textbf\def\PY@tc##1{\textcolor[rgb]{0.00,0.50,0.00}{##1}}}
\@namedef{PY@tok@nd}{\def\PY@tc##1{\textcolor[rgb]{0.67,0.13,1.00}{##1}}}
\@namedef{PY@tok@s}{\def\PY@tc##1{\textcolor[rgb]{0.73,0.13,0.13}{##1}}}
\@namedef{PY@tok@sd}{\let\PY@it=\textit\def\PY@tc##1{\textcolor[rgb]{0.73,0.13,0.13}{##1}}}
\@namedef{PY@tok@si}{\let\PY@bf=\textbf\def\PY@tc##1{\textcolor[rgb]{0.64,0.35,0.47}{##1}}}
\@namedef{PY@tok@se}{\let\PY@bf=\textbf\def\PY@tc##1{\textcolor[rgb]{0.67,0.36,0.12}{##1}}}
\@namedef{PY@tok@sr}{\def\PY@tc##1{\textcolor[rgb]{0.64,0.35,0.47}{##1}}}
\@namedef{PY@tok@ss}{\def\PY@tc##1{\textcolor[rgb]{0.10,0.09,0.49}{##1}}}
\@namedef{PY@tok@sx}{\def\PY@tc##1{\textcolor[rgb]{0.00,0.50,0.00}{##1}}}
\@namedef{PY@tok@m}{\def\PY@tc##1{\textcolor[rgb]{0.40,0.40,0.40}{##1}}}
\@namedef{PY@tok@gh}{\let\PY@bf=\textbf\def\PY@tc##1{\textcolor[rgb]{0.00,0.00,0.50}{##1}}}
\@namedef{PY@tok@gu}{\let\PY@bf=\textbf\def\PY@tc##1{\textcolor[rgb]{0.50,0.00,0.50}{##1}}}
\@namedef{PY@tok@gd}{\def\PY@tc##1{\textcolor[rgb]{0.63,0.00,0.00}{##1}}}
\@namedef{PY@tok@gi}{\def\PY@tc##1{\textcolor[rgb]{0.00,0.52,0.00}{##1}}}
\@namedef{PY@tok@gr}{\def\PY@tc##1{\textcolor[rgb]{0.89,0.00,0.00}{##1}}}
\@namedef{PY@tok@ge}{\let\PY@it=\textit}
\@namedef{PY@tok@gs}{\let\PY@bf=\textbf}
\@namedef{PY@tok@gp}{\let\PY@bf=\textbf\def\PY@tc##1{\textcolor[rgb]{0.00,0.00,0.50}{##1}}}
\@namedef{PY@tok@go}{\def\PY@tc##1{\textcolor[rgb]{0.44,0.44,0.44}{##1}}}
\@namedef{PY@tok@gt}{\def\PY@tc##1{\textcolor[rgb]{0.00,0.27,0.87}{##1}}}
\@namedef{PY@tok@err}{\def\PY@bc##1{{\setlength{\fboxsep}{\string -\fboxrule}\fcolorbox[rgb]{1.00,0.00,0.00}{1,1,1}{\strut ##1}}}}
\@namedef{PY@tok@kc}{\let\PY@bf=\textbf\def\PY@tc##1{\textcolor[rgb]{0.00,0.50,0.00}{##1}}}
\@namedef{PY@tok@kd}{\let\PY@bf=\textbf\def\PY@tc##1{\textcolor[rgb]{0.00,0.50,0.00}{##1}}}
\@namedef{PY@tok@kn}{\let\PY@bf=\textbf\def\PY@tc##1{\textcolor[rgb]{0.00,0.50,0.00}{##1}}}
\@namedef{PY@tok@kr}{\let\PY@bf=\textbf\def\PY@tc##1{\textcolor[rgb]{0.00,0.50,0.00}{##1}}}
\@namedef{PY@tok@bp}{\def\PY@tc##1{\textcolor[rgb]{0.00,0.50,0.00}{##1}}}
\@namedef{PY@tok@fm}{\def\PY@tc##1{\textcolor[rgb]{0.00,0.00,1.00}{##1}}}
\@namedef{PY@tok@vc}{\def\PY@tc##1{\textcolor[rgb]{0.10,0.09,0.49}{##1}}}
\@namedef{PY@tok@vg}{\def\PY@tc##1{\textcolor[rgb]{0.10,0.09,0.49}{##1}}}
\@namedef{PY@tok@vi}{\def\PY@tc##1{\textcolor[rgb]{0.10,0.09,0.49}{##1}}}
\@namedef{PY@tok@vm}{\def\PY@tc##1{\textcolor[rgb]{0.10,0.09,0.49}{##1}}}
\@namedef{PY@tok@sa}{\def\PY@tc##1{\textcolor[rgb]{0.73,0.13,0.13}{##1}}}
\@namedef{PY@tok@sb}{\def\PY@tc##1{\textcolor[rgb]{0.73,0.13,0.13}{##1}}}
\@namedef{PY@tok@sc}{\def\PY@tc##1{\textcolor[rgb]{0.73,0.13,0.13}{##1}}}
\@namedef{PY@tok@dl}{\def\PY@tc##1{\textcolor[rgb]{0.73,0.13,0.13}{##1}}}
\@namedef{PY@tok@s2}{\def\PY@tc##1{\textcolor[rgb]{0.73,0.13,0.13}{##1}}}
\@namedef{PY@tok@sh}{\def\PY@tc##1{\textcolor[rgb]{0.73,0.13,0.13}{##1}}}
\@namedef{PY@tok@s1}{\def\PY@tc##1{\textcolor[rgb]{0.73,0.13,0.13}{##1}}}
\@namedef{PY@tok@mb}{\def\PY@tc##1{\textcolor[rgb]{0.40,0.40,0.40}{##1}}}
\@namedef{PY@tok@mf}{\def\PY@tc##1{\textcolor[rgb]{0.40,0.40,0.40}{##1}}}
\@namedef{PY@tok@mh}{\def\PY@tc##1{\textcolor[rgb]{0.40,0.40,0.40}{##1}}}
\@namedef{PY@tok@mi}{\def\PY@tc##1{\textcolor[rgb]{0.40,0.40,0.40}{##1}}}
\@namedef{PY@tok@il}{\def\PY@tc##1{\textcolor[rgb]{0.40,0.40,0.40}{##1}}}
\@namedef{PY@tok@mo}{\def\PY@tc##1{\textcolor[rgb]{0.40,0.40,0.40}{##1}}}
\@namedef{PY@tok@ch}{\let\PY@it=\textit\def\PY@tc##1{\textcolor[rgb]{0.24,0.48,0.48}{##1}}}
\@namedef{PY@tok@cm}{\let\PY@it=\textit\def\PY@tc##1{\textcolor[rgb]{0.24,0.48,0.48}{##1}}}
\@namedef{PY@tok@cpf}{\let\PY@it=\textit\def\PY@tc##1{\textcolor[rgb]{0.24,0.48,0.48}{##1}}}
\@namedef{PY@tok@c1}{\let\PY@it=\textit\def\PY@tc##1{\textcolor[rgb]{0.24,0.48,0.48}{##1}}}
\@namedef{PY@tok@cs}{\let\PY@it=\textit\def\PY@tc##1{\textcolor[rgb]{0.24,0.48,0.48}{##1}}}

\def\PYZbs{\char`\\}
\def\PYZus{\char`\_}
\def\PYZob{\char`\{}
\def\PYZcb{\char`\}}
\def\PYZca{\char`\^}
\def\PYZam{\char`\&}
\def\PYZlt{\char`\<}
\def\PYZgt{\char`\>}
\def\PYZsh{\char`\#}
\def\PYZpc{\char`\%}
\def\PYZdl{\char`\$}
\def\PYZhy{\char`\-}
\def\PYZsq{\char`\'}
\def\PYZdq{\char`\"}
\def\PYZti{\char`\~}
% for compatibility with earlier versions
\def\PYZat{@}
\def\PYZlb{[}
\def\PYZrb{]}
\makeatother


    % For linebreaks inside Verbatim environment from package fancyvrb.
    \makeatletter
        \newbox\Wrappedcontinuationbox
        \newbox\Wrappedvisiblespacebox
        \newcommand*\Wrappedvisiblespace {\textcolor{red}{\textvisiblespace}}
        \newcommand*\Wrappedcontinuationsymbol {\textcolor{red}{\llap{\tiny$\m@th\hookrightarrow$}}}
        \newcommand*\Wrappedcontinuationindent {3ex }
        \newcommand*\Wrappedafterbreak {\kern\Wrappedcontinuationindent\copy\Wrappedcontinuationbox}
        % Take advantage of the already applied Pygments mark-up to insert
        % potential linebreaks for TeX processing.
        %        {, <, #, %, $, ' and ": go to next line.
        %        _, }, ^, &, >, - and ~: stay at end of broken line.
        % Use of \textquotesingle for straight quote.
        \newcommand*\Wrappedbreaksatspecials {%
            \def\PYGZus{\discretionary{\char`\_}{\Wrappedafterbreak}{\char`\_}}%
            \def\PYGZob{\discretionary{}{\Wrappedafterbreak\char`\{}{\char`\{}}%
            \def\PYGZcb{\discretionary{\char`\}}{\Wrappedafterbreak}{\char`\}}}%
            \def\PYGZca{\discretionary{\char`\^}{\Wrappedafterbreak}{\char`\^}}%
            \def\PYGZam{\discretionary{\char`\&}{\Wrappedafterbreak}{\char`\&}}%
            \def\PYGZlt{\discretionary{}{\Wrappedafterbreak\char`\<}{\char`\<}}%
            \def\PYGZgt{\discretionary{\char`\>}{\Wrappedafterbreak}{\char`\>}}%
            \def\PYGZsh{\discretionary{}{\Wrappedafterbreak\char`\#}{\char`\#}}%
            \def\PYGZpc{\discretionary{}{\Wrappedafterbreak\char`\%}{\char`\%}}%
            \def\PYGZdl{\discretionary{}{\Wrappedafterbreak\char`\$}{\char`\$}}%
            \def\PYGZhy{\discretionary{\char`\-}{\Wrappedafterbreak}{\char`\-}}%
            \def\PYGZsq{\discretionary{}{\Wrappedafterbreak\textquotesingle}{\textquotesingle}}%
            \def\PYGZdq{\discretionary{}{\Wrappedafterbreak\char`\"}{\char`\"}}%
            \def\PYGZti{\discretionary{\char`\~}{\Wrappedafterbreak}{\char`\~}}%
        }
        % Some characters . , ; ? ! / are not pygmentized.
        % This macro makes them "active" and they will insert potential linebreaks
        \newcommand*\Wrappedbreaksatpunct {%
            \lccode`\~`\.\lowercase{\def~}{\discretionary{\hbox{\char`\.}}{\Wrappedafterbreak}{\hbox{\char`\.}}}%
            \lccode`\~`\,\lowercase{\def~}{\discretionary{\hbox{\char`\,}}{\Wrappedafterbreak}{\hbox{\char`\,}}}%
            \lccode`\~`\;\lowercase{\def~}{\discretionary{\hbox{\char`\;}}{\Wrappedafterbreak}{\hbox{\char`\;}}}%
            \lccode`\~`\:\lowercase{\def~}{\discretionary{\hbox{\char`\:}}{\Wrappedafterbreak}{\hbox{\char`\:}}}%
            \lccode`\~`\?\lowercase{\def~}{\discretionary{\hbox{\char`\?}}{\Wrappedafterbreak}{\hbox{\char`\?}}}%
            \lccode`\~`\!\lowercase{\def~}{\discretionary{\hbox{\char`\!}}{\Wrappedafterbreak}{\hbox{\char`\!}}}%
            \lccode`\~`\/\lowercase{\def~}{\discretionary{\hbox{\char`\/}}{\Wrappedafterbreak}{\hbox{\char`\/}}}%
            \catcode`\.\active
            \catcode`\,\active
            \catcode`\;\active
            \catcode`\:\active
            \catcode`\?\active
            \catcode`\!\active
            \catcode`\/\active
            \lccode`\~`\~
        }
    \makeatother

    \let\OriginalVerbatim=\Verbatim
    \makeatletter
    \renewcommand{\Verbatim}[1][1]{%
        %\parskip\z@skip
        \sbox\Wrappedcontinuationbox {\Wrappedcontinuationsymbol}%
        \sbox\Wrappedvisiblespacebox {\FV@SetupFont\Wrappedvisiblespace}%
        \def\FancyVerbFormatLine ##1{\hsize\linewidth
            \vtop{\raggedright\hyphenpenalty\z@\exhyphenpenalty\z@
                \doublehyphendemerits\z@\finalhyphendemerits\z@
                \strut ##1\strut}%
        }%
        % If the linebreak is at a space, the latter will be displayed as visible
        % space at end of first line, and a continuation symbol starts next line.
        % Stretch/shrink are however usually zero for typewriter font.
        \def\FV@Space {%
            \nobreak\hskip\z@ plus\fontdimen3\font minus\fontdimen4\font
            \discretionary{\copy\Wrappedvisiblespacebox}{\Wrappedafterbreak}
            {\kern\fontdimen2\font}%
        }%

        % Allow breaks at special characters using \PYG... macros.
        \Wrappedbreaksatspecials
        % Breaks at punctuation characters . , ; ? ! and / need catcode=\active
        \OriginalVerbatim[#1,codes*=\Wrappedbreaksatpunct]%
    }
    \makeatother

    % Exact colors from NB
    \definecolor{incolor}{HTML}{303F9F}
    \definecolor{outcolor}{HTML}{D84315}
    \definecolor{cellborder}{HTML}{CFCFCF}
    \definecolor{cellbackground}{HTML}{F7F7F7}

    % prompt
    \makeatletter
    \newcommand{\boxspacing}{\kern\kvtcb@left@rule\kern\kvtcb@boxsep}
    \makeatother
    \newcommand{\prompt}[4]{
        {\ttfamily\llap{{\color{#2}[#3]:\hspace{3pt}#4}}\vspace{-\baselineskip}}
    }
    

    
    % Start the section counter at -1, so the Table of Contents is Section 0
    \setcounter{section}{-2}
    % Prevent overflowing lines due to hard-to-break entities
    \sloppy
    % Setup hyperref package
    \hypersetup{
      breaklinks=true,  % so long urls are correctly broken across lines
      colorlinks=true,
      urlcolor=urlcolor,
      linkcolor=linkcolor,
      citecolor=citecolor,
      }
    % Slightly smaller margins than the latex defaults
    
    \geometry{verbose,tmargin=0.5in,bmargin=0.5in,lmargin=0.5in,rmargin=0.5in}
    
    

\begin{document}
    
    \maketitle
    
    

    
    \hypertarget{validating-runge-kutta-butcher-tables-using-truncated-taylor-series}{%
\section{Validating Runge Kutta Butcher tables using Truncated Taylor
Series}\label{validating-runge-kutta-butcher-tables-using-truncated-taylor-series}}

\hypertarget{authors-zach-etienne-brandon-clark-gabriel-m-steward}{%
\subsection{Authors: Zach Etienne, Brandon Clark, \& Gabriel M
Steward}\label{authors-zach-etienne-brandon-clark-gabriel-m-steward}}

\hypertarget{this-tutorial-notebook-is-designed-to-validate-the-butcher-tables-contained-within-the-butcher-dictionary-constructed-in-the-rk-butcher-table-dictionary-nrpy-module.}{%
\subsection{\texorpdfstring{This tutorial notebook is designed to
validate the Butcher tables contained within the Butcher dictionary
constructed in the \href{Tutorial-RK_Butcher_Table_Dictionary.ipynb}{RK
Butcher Table Dictionary} NRPy+
module.}{This tutorial notebook is designed to validate the Butcher tables contained within the Butcher dictionary constructed in the RK Butcher Table Dictionary NRPy+ module.}}\label{this-tutorial-notebook-is-designed-to-validate-the-butcher-tables-contained-within-the-butcher-dictionary-constructed-in-the-rk-butcher-table-dictionary-nrpy-module.}}

\hypertarget{nrpy-source-code-for-this-module}{%
\subsubsection{NRPy+ Source Code for this
module:}\label{nrpy-source-code-for-this-module}}

\begin{itemize}
\tightlist
\item
  \href{../edit/MoLtimestepping/RK_Butcher_Table_Validation.py}{MoLtimestepping/RK\_Butcher\_Table\_Validation.py}
  stores the \texttt{Validate} function for validating convergence
  orders for Runge Kutta methods.
\item
  \href{../edit/MoLtimestepping/RK_Butcher_Table_Dictionary.py}{MoLtimestepping/RK\_Butcher\_Table\_Dictionary.py}
  \href{Tutorial-RK_Butcher_Table_Dictionary.ipynb}{{[}\textbf{tutorial}{]}}
  accesses the Butcher table dictionary \texttt{Butcher\_dict} for known
  explicit Runge Kutta methods.
\end{itemize}

\hypertarget{introduction}{%
\subsection{Introduction:}\label{introduction}}

Starting with the ODE (ordinary differential equation) initial value
problem: \[
y'(t) = f(y,t)\ \ \ y\left(t=0\right)=y_0,
\] for various choices of \(f(y,t)\), this module validates the Runge
Kutta (RK) methods coded in
\href{../edit/MoLtimestepping/RK_Butcher_Table_Dictionary.py}{RK\_Butcher\_Table\_Dictionary.py}
\href{Tutorial-RK_Butcher_Table_Dictionary.ipynb}{\textbf{tutorial
notebook}} as follows.

Given \(y_0\) and a smooth \(f(y,t)\), all explicit RK methods provide
an estimate for \(y_1 = y\left(\Delta t\right)\), with an error term
that is proportional to \(\left(\Delta t\right)^m\), where \(m\) is an
integer typically greater than zero. This error term corresponds to the
\emph{local} truncation error. For RK4, for example, while the
\emph{total accumulated truncation error} (i.e., the accumulated error
at a fixed final time \(t_f\)) is proportional to
\(\left(\Delta t\right)^4\), the \emph{local} truncation error (i.e.,
the error after one arbitrarily chosen timestep \(\Delta t\)) is
proportional to \(\left(\Delta t\right)^5\).

If the exact solution \(y(t)\) is known as a closed-form expression,
then \(y\left(\Delta t\right)\) can be \emph{separately} written as a
Taylor expansion about \(y(t=0)\):

\[
y\left(\Delta t\right) = \sum_{n=0}^\infty \frac{y^{(n)}(t=0)}{n!} \left(\Delta t\right)^n,
\] where \(y^{(n)}(t=0)\) is the \(n\)th derivative of \(y(t)\)
evaluated at \(t=0\).

The above expression will be known exactly. Furthermore, if one chooses
a numerical value for \(y_0\) \emph{and leaves \(\Delta t\)
unspecified}, any explicit RK method will provide an estimate for
\(y\left(\Delta t\right)\) of the form

\[
y\left(\Delta t\right) = \sum_{n=0}^\infty a_n \left(\Delta t\right)^n,
\] where \(a_n\) \emph{must} match the Taylor expansion of the
\emph{exact} solution at least up to and including terms proportional to
\(\left(\Delta t\right)^m\), where \(m\) is the order of the local
truncation error. If this is \emph{not} the case, then the Butcher table
was almost certainly \emph{not} typed correctly.

Therefore, comparing the numerical result with unspecified \(\Delta t\)
against the exact Taylor series provides a convenient (though not
perfectly robust) means to verify that the Butcher table for a given RK
method was typed correctly. Multiple typos in the Butcher tables were
found using this approach.

\textbf{Example from Z. Etienne's MATH 521 (Numerical Analysis) lecture
notes:}

Consider the ODE \[
y' = y - 2 t e^{-2t},\quad y(0)=y(t_0)=0.
\]

\begin{itemize}
\item
  Solve this ODE exactly, then Taylor expand the solution about \(t=0\)
  to approximate the solution at \(y(t=\Delta t)\) to fifth order in
  \(\Delta t\).
\item
  Next, solve this ODE using Heun's method (second order in total
  accumulated truncation error, third order in local truncation error)
  \{\it by hand\} with a step size of \(\Delta t\) to find
  \(y(\Delta t)\). Confirm that the solution obtained when using Heun's
  method has an error term that is at worst
  \(\mathcal{O}\left((\Delta t)^3\right)\). If the dominant error is
  proportional to a higher power of \(\Delta t\), explain the
  discrepancy.
\item
  Finally, solve this ODE using the Ralston method \{\it by hand\} with
  a step size of \(\Delta t\) to find \(y(\Delta t)\). Is the
  coefficient on the dominant error term closer to the exact solution
  than Heun's method?
\end{itemize}

We can solve this equation via the method of integrating factors, which
states that ODEs of the form: \[
y'(t) + p(t) y(t) = g(t)
\] are solved via \[
y(t) = \frac{1}{\mu(t)} \left[ \int \mu(s) g(s) ds + c \right],
\] where the integrating factor \(\mu(t)\) is given by \[
\mu(t) = \exp\left(\int p(t) dt\right).
\]

Here, \(p(t)=-1\) and \(g(t) = - 2 t e^{-2t}\). Then

\begin{equation}
\mu(t) = \exp\left(-\int dt\right) = e^{-t+c} = k e^{-t}
\end{equation} and

\begin{align}
y(t) &= e^t/k  \left[ \int k e^{-s} (- 2 s e^{-2s}) ds + c \right] = -2 e^t \left[ \int s e^{-3s} ds + c' \right] \\
&= -2 e^t \left[ e^{-3 t} \left(-\frac{t}{3}-\frac{1}{9}\right) + c' \right] = -2 e^{-2t} \left(-\frac{t}{3}-\frac{1}{9}\right) -2 c' e^t \\
&= e^{-2t} \left(2\frac{t}{3}+\frac{2}{9}\right) + c'' e^t. \\
\end{align}

If \(y(0)=0\) then we can compute the integration constant \(c''\), and
\(y(t)\) becomes \[
y(t) = \frac{2}{9} e^{-2 t} \left(3 t + 1 - e^{3 t}\right).
\]

The Taylor Series expansion of the exact solution about \(t=0\)
evaluated at \(y(\Delta t)\) yields \[
y(\Delta t) = -(\Delta t)^2+(\Delta t)^3-\frac{3 (\Delta t)^4}{4}+\frac{23 (\Delta
  t)^5}{60}-\frac{19 (\Delta t)^6}{120}+O\left((\Delta t)^7\right).
\]

Next we evaluate \(y(\Delta t)\) using Heun's method. We know
\(y(0)=y_0=0\) and \(f(y,t)=y - 2 t e^{-2t}\), so \begin{align}
k_1 &= \Delta t f(y(0),0) \\
    &= \Delta t \times 0 \\
    &= 0 \\
k_2 &= \Delta t f(y(0)+k_1,0+\Delta t) \\
   &= \Delta t f(y(0)+0,0+\Delta t) \\
   &= \Delta t (-2 \Delta t e^{-2\Delta t}) \\
   &= -2 (\Delta t)^2 e^{-2\Delta t} \\
y(\Delta t) &= y_0 + \frac{1}{2} (k_1 + k_2) + \mathcal{O}\left((\Delta t)^3\right) \\
&= 0 - (\Delta t)^2 e^{-2\Delta t} \\
&= - (\Delta t)^2 ( 1 - 2 \Delta t + 2 (\Delta t)^2 + ...) \\
&= - (\Delta t)^2 + 2 (\Delta t)^3 + \mathcal{O}\left((\Delta t)^4\right).
\end{align}

Thus the coefficient on the \((\Delta t)^3\) term is wrong, but this is
completely consistent with the fact that our stepping scheme is only
third-order accurate in \(\Delta t\).

In the below approach, the RK result is subtracted from the exact Taylor
series result, as a check to determine whether the RK Butcher table was
coded correctly; if it was not, then the odds are good that the RK
results will not match to the expected local truncation error order.
Multiple \(f(y,t)\) are coded below to improve the robustness of this
test.

As NRPy+'s butcher tables also contain the information to run
Adams-Bashforth (AB) methods, those too shall also be validated.
Fortunately the exact same validation method that we use for the RK
methods also functions for the AB methods, all that needs to change is
the specific implementation code.

    \hypertarget{table-of-contents}{%
\section{Table of Contents}\label{table-of-contents}}

\[\label{toc}\]

This notebook is organized as follows

\begin{enumerate}
\def\labelenumi{\arabic{enumi}.}
\tightlist
\item
  \hyperref[initializenrpy]{Step 1}: Initialize needed Python/NRPy+
  modules
\item
  \hyperref[table_validate]{Step 2} Validate Convergence Order of
  Butcher Tables

  \begin{enumerate}
  \def\labelenumii{\arabic{enumii}.}
  \tightlist
  \item
    \hyperref[rhs]{Step 2.a}: Defining the right-hand side of the ODE
  \item
    \hyperref[validfunc]{Step 2.b}: Defining a Validation Function
  \item
    \hyperref[rkvalid]{Step 2.c}: Validating RK Methods against ODEs
  \item
    \hyperref[arkvalid]{Step 2.d}: Validating Inherently Adaptive RK
    Methods against ODEs
  \item
    \hyperref[abvalid]{Step 2.e}: Validating AB Methods against ODEs
  \end{enumerate}
\item
  \hyperref[latex_pdf_output]{Step 3}: Output this notebook to
  \(\LaTeX\)-formatted PDF file
\end{enumerate}

    \hypertarget{step-1-initialize-needed-pythonnrpy-modules-back-to-top}{%
\section{\texorpdfstring{Step 1: Initialize needed Python/NRPy+ modules
{[}Back to
\hyperref[toc]{top}{]}}{Step 1: Initialize needed Python/NRPy+ modules {[}Back to {]}}}\label{step-1-initialize-needed-pythonnrpy-modules-back-to-top}}

\[\label{initializenrpy}\]

Let's start by importing all the needed modules from Python/NRPy+:

    \begin{tcolorbox}[breakable, size=fbox, boxrule=1pt, pad at break*=1mm,colback=cellbackground, colframe=cellborder]
\prompt{In}{incolor}{1}{\boxspacing}
\begin{Verbatim}[commandchars=\\\{\}]
\PY{k+kn}{import} \PY{n+nn}{sympy} \PY{k}{as} \PY{n+nn}{sp}              \PY{c+c1}{\PYZsh{} SymPy: The Python computer algebra package upon which NRPy+ depends}
\PY{k+kn}{import} \PY{n+nn}{numpy} \PY{k}{as} \PY{n+nn}{np}              \PY{c+c1}{\PYZsh{} NumPy: A numerical methods module for Python}
\PY{k+kn}{from} \PY{n+nn}{MoLtimestepping}\PY{n+nn}{.}\PY{n+nn}{RK\PYZus{}Butcher\PYZus{}Table\PYZus{}Dictionary} \PY{k+kn}{import} \PY{n}{Butcher\PYZus{}dict}
\end{Verbatim}
\end{tcolorbox}

    \hypertarget{step-2-validate-convergence-order-of-butcher-tables-back-to-top}{%
\section{\texorpdfstring{Step 2: Validate Convergence Order of Butcher
Tables {[}Back to
\hyperref[toc]{top}{]}}{Step 2: Validate Convergence Order of Butcher Tables {[}Back to {]}}}\label{step-2-validate-convergence-order-of-butcher-tables-back-to-top}}

\[\label{table_validate}\]

Each Butcher table/Runge Kutta method is tested by solving an ODE.
Comparing the Taylor series expansions of the exact solution and the
numerical solution as discussed in the \textbf{Introduction} above will
confirm whether the method converges to the appropriate order.

    \hypertarget{step-2.a-defining-the-right-hand-side-of-the-ode-back-to-top}{%
\subsection{\texorpdfstring{Step 2.a: Defining the right-hand side of
the ODE {[}Back to
\hyperref[toc]{top}{]}}{Step 2.a: Defining the right-hand side of the ODE {[}Back to {]}}}\label{step-2.a-defining-the-right-hand-side-of-the-ode-back-to-top}}

\[\label{rhs}\]

Consider the form of ODE \(y'=f(y,t)\). The following begins to
construct a dictionary \texttt{rhs\_dict} of right-hand side functions
for us to validate explicit Runge Kutta methods. The most up-to-date
catalog of functions stored in \texttt{rhs\_dict} can be found in the
\href{../edit/MoLtimestepping/RK_Butcher_Table_Validation.py}{RK\_Butcher\_Table\_Validation.py}
module.

    \begin{tcolorbox}[breakable, size=fbox, boxrule=1pt, pad at break*=1mm,colback=cellbackground, colframe=cellborder]
\prompt{In}{incolor}{2}{\boxspacing}
\begin{Verbatim}[commandchars=\\\{\}]
\PY{k}{def} \PY{n+nf}{fypt}\PY{p}{(}\PY{n}{y}\PY{p}{,}\PY{n}{t}\PY{p}{)}\PY{p}{:} \PY{c+c1}{\PYZsh{} Yields expected convergence order for all cases}
    \PY{c+c1}{\PYZsh{}            except DP6 which converge to higher order (7, respectively)}
    \PY{k}{return} \PY{n}{y}\PY{o}{+}\PY{n}{t}

\PY{k}{def} \PY{n+nf}{fy}\PY{p}{(}\PY{n}{y}\PY{p}{,}\PY{n}{t}\PY{p}{)}\PY{p}{:} \PY{c+c1}{\PYZsh{} Yields expected convergence order for all cases}
    \PY{k}{return} \PY{n}{y}

\PY{k}{def} \PY{n+nf}{feypt}\PY{p}{(}\PY{n}{y}\PY{p}{,}\PY{n}{t}\PY{p}{)}\PY{p}{:} \PY{c+c1}{\PYZsh{} Yields expected convergence order for all cases}
    \PY{k}{return} \PY{n}{sp}\PY{o}{.}\PY{n}{exp}\PY{p}{(}\PY{l+m+mf}{1.0}\PY{o}{*}\PY{p}{(}\PY{n}{y}\PY{o}{+}\PY{n}{t}\PY{p}{)}\PY{p}{)}

\PY{k}{def} \PY{n+nf}{ftpoly6}\PY{p}{(}\PY{n}{y}\PY{p}{,}\PY{n}{t}\PY{p}{)}\PY{p}{:} \PY{c+c1}{\PYZsh{} Yields expected convergence order for all cases, L6 has 0 error}
    \PY{k}{return} \PY{l+m+mi}{2}\PY{o}{*}\PY{n}{t}\PY{o}{*}\PY{o}{*}\PY{l+m+mi}{6}\PY{o}{\PYZhy{}}\PY{l+m+mi}{389}\PY{o}{*}\PY{n}{t}\PY{o}{*}\PY{o}{*}\PY{l+m+mi}{5}\PY{o}{+}\PY{l+m+mi}{15}\PY{o}{*}\PY{n}{t}\PY{o}{*}\PY{o}{*}\PY{l+m+mi}{4}\PY{o}{\PYZhy{}}\PY{l+m+mi}{22}\PY{o}{*}\PY{n}{t}\PY{o}{*}\PY{o}{*}\PY{l+m+mi}{3}\PY{o}{+}\PY{l+m+mi}{81}\PY{o}{*}\PY{n}{t}\PY{o}{*}\PY{o}{*}\PY{l+m+mi}{2}\PY{o}{\PYZhy{}}\PY{n}{t}\PY{o}{+}\PY{l+m+mi}{42}
\PY{n}{rhs\PYZus{}dict} \PY{o}{=} \PY{p}{\PYZob{}}\PY{l+s+s1}{\PYZsq{}}\PY{l+s+s1}{ypt}\PY{l+s+s1}{\PYZsq{}}\PY{p}{:}\PY{n}{fypt}\PY{p}{,} \PY{l+s+s1}{\PYZsq{}}\PY{l+s+s1}{y}\PY{l+s+s1}{\PYZsq{}}\PY{p}{:}\PY{n}{fy}\PY{p}{,} \PY{l+s+s1}{\PYZsq{}}\PY{l+s+s1}{eypt}\PY{l+s+s1}{\PYZsq{}}\PY{p}{:}\PY{n}{feypt}\PY{p}{,} \PY{l+s+s1}{\PYZsq{}}\PY{l+s+s1}{tpoly6}\PY{l+s+s1}{\PYZsq{}}\PY{p}{:}\PY{n}{ftpoly6}\PY{p}{\PYZcb{}}
\end{Verbatim}
\end{tcolorbox}

    \hypertarget{step-2.b-defining-a-validation-function-back-to-top}{%
\subsection{\texorpdfstring{Step 2.b: Defining a Validation Function
{[}Back to
\hyperref[toc]{top}{]}}{Step 2.b: Defining a Validation Function {[}Back to {]}}}\label{step-2.b-defining-a-validation-function-back-to-top}}

\[\label{validfunc}\]

To validate each Butcher table we compare the exact solutions to ODEs
with the numerical solutions using the Runge Kutta scheme built into
each Butcher table. The following is a function that

\begin{enumerate}
\def\labelenumi{\arabic{enumi}.}
\tightlist
\item
  solves the ODE exactly,
\item
  solves the ODE numerically for a given Butcher table, and
\item
  compares the two solutions and checks for the order of convergence by
  returning their difference.
\end{enumerate}

The \texttt{Validate()} function inputs a specified
\texttt{Butcher\_key}, the starting guess solution and time
\texttt{y\_n}, \texttt{t\_n} and the right-hand side of the ODE
corresponding to a specified initial value problem, \texttt{rhs\_key}.

    \begin{tcolorbox}[breakable, size=fbox, boxrule=1pt, pad at break*=1mm,colback=cellbackground, colframe=cellborder]
\prompt{In}{incolor}{3}{\boxspacing}
\begin{Verbatim}[commandchars=\\\{\}]
\PY{k+kn}{from} \PY{n+nn}{MoLtimestepping}\PY{n+nn}{.}\PY{n+nn}{RK\PYZus{}Butcher\PYZus{}Table\PYZus{}Dictionary} \PY{k+kn}{import} \PY{n}{Butcher\PYZus{}dict}
\PY{k}{def} \PY{n+nf}{Validate}\PY{p}{(}\PY{n}{Butcher\PYZus{}key}\PY{p}{,} \PY{n}{yn}\PY{p}{,} \PY{n}{tn}\PY{p}{,} \PY{n}{rhs\PYZus{}key}\PY{p}{)}\PY{p}{:}
    \PY{c+c1}{\PYZsh{} 1. First we solve the ODE exactly}
    \PY{n}{y} \PY{o}{=} \PY{n}{sp}\PY{o}{.}\PY{n}{Function}\PY{p}{(}\PY{l+s+s1}{\PYZsq{}}\PY{l+s+s1}{y}\PY{l+s+s1}{\PYZsq{}}\PY{p}{)}
    \PY{n}{sol} \PY{o}{=} \PY{n}{sp}\PY{o}{.}\PY{n}{dsolve}\PY{p}{(}\PY{n}{sp}\PY{o}{.}\PY{n}{Eq}\PY{p}{(}\PY{n}{y}\PY{p}{(}\PY{n}{t}\PY{p}{)}\PY{o}{.}\PY{n}{diff}\PY{p}{(}\PY{n}{t}\PY{p}{)}\PY{p}{,} \PY{n}{rhs\PYZus{}dict}\PY{p}{[}\PY{n}{rhs\PYZus{}key}\PY{p}{]}\PY{p}{(}\PY{n}{y}\PY{p}{(}\PY{n}{t}\PY{p}{)}\PY{p}{,} \PY{n}{t}\PY{p}{)}\PY{p}{)}\PY{p}{,} \PY{n}{y}\PY{p}{(}\PY{n}{t}\PY{p}{)}\PY{p}{)}\PY{o}{.}\PY{n}{rhs}
    \PY{n}{constants} \PY{o}{=} \PY{n}{sp}\PY{o}{.}\PY{n}{solve}\PY{p}{(}\PY{p}{[}\PY{n}{sol}\PY{o}{.}\PY{n}{subs}\PY{p}{(}\PY{n}{t}\PY{p}{,}\PY{n}{tn}\PY{p}{)}\PY{o}{\PYZhy{}}\PY{n}{yn}\PY{p}{]}\PY{p}{)}
    \PY{n}{exact} \PY{o}{=} \PY{n}{sol}\PY{o}{.}\PY{n}{subs}\PY{p}{(}\PY{n}{constants}\PY{p}{)}

    \PY{c+c1}{\PYZsh{} 2. Now we solve the ODE numerically using specified Butcher table}

    \PY{c+c1}{\PYZsh{} Access the requested Butcher table}
    \PY{n}{Butcher} \PY{o}{=} \PY{n}{Butcher\PYZus{}dict}\PY{p}{[}\PY{n}{Butcher\PYZus{}key}\PY{p}{]}\PY{p}{[}\PY{l+m+mi}{0}\PY{p}{]}
    \PY{c+c1}{\PYZsh{} Determine number of predictor\PYZhy{}corrector steps}
    \PY{n}{L} \PY{o}{=} \PY{n+nb}{len}\PY{p}{(}\PY{n}{Butcher}\PY{p}{)}\PY{o}{\PYZhy{}}\PY{l+m+mi}{1}
    \PY{c+c1}{\PYZsh{} Set a temporary array for update values}
    \PY{n}{k} \PY{o}{=} \PY{n}{np}\PY{o}{.}\PY{n}{zeros}\PY{p}{(}\PY{n}{L}\PY{p}{,} \PY{n}{dtype}\PY{o}{=}\PY{n+nb}{object}\PY{p}{)}
    \PY{c+c1}{\PYZsh{} Initialize intermediate variable}
    \PY{n}{yhat} \PY{o}{=} \PY{l+m+mi}{0}
    \PY{c+c1}{\PYZsh{} Initialize the updated solution}
    \PY{n}{ynp1} \PY{o}{=} \PY{l+m+mi}{0}
    \PY{k}{for} \PY{n}{i} \PY{o+ow}{in} \PY{n+nb}{range}\PY{p}{(}\PY{n}{L}\PY{p}{)}\PY{p}{:}
        \PY{c+c1}{\PYZsh{}Initialize and approximate update for solution}
        \PY{n}{yhat} \PY{o}{=} \PY{n}{yn}
        \PY{k}{for} \PY{n}{j} \PY{o+ow}{in} \PY{n+nb}{range}\PY{p}{(}\PY{n}{i}\PY{p}{)}\PY{p}{:}
            \PY{c+c1}{\PYZsh{} Update yhat for solution using a\PYZus{}ij Butcher table coefficients}
            \PY{n}{yhat} \PY{o}{+}\PY{o}{=} \PY{n}{Butcher}\PY{p}{[}\PY{n}{i}\PY{p}{]}\PY{p}{[}\PY{n}{j}\PY{o}{+}\PY{l+m+mi}{1}\PY{p}{]}\PY{o}{*}\PY{n}{k}\PY{p}{[}\PY{n}{j}\PY{p}{]}
            \PY{k}{if} \PY{n}{Butcher\PYZus{}key} \PY{o}{==} \PY{l+s+s2}{\PYZdq{}}\PY{l+s+s2}{DP8}\PY{l+s+s2}{\PYZdq{}} \PY{o+ow}{or} \PY{n}{Butcher\PYZus{}key} \PY{o}{==} \PY{l+s+s2}{\PYZdq{}}\PY{l+s+s2}{L6}\PY{l+s+s2}{\PYZdq{}}\PY{p}{:}
                \PY{n}{yhat} \PY{o}{=} \PY{l+m+mf}{1.0}\PY{o}{*}\PY{n}{sp}\PY{o}{.}\PY{n}{N}\PY{p}{(}\PY{n}{yhat}\PY{p}{,}\PY{l+m+mi}{20}\PY{p}{)} \PY{c+c1}{\PYZsh{} Otherwise the adding of fractions kills performance.}
        \PY{c+c1}{\PYZsh{} Determine the next corrector variable k\PYZus{}i using c\PYZus{}i Butcher table coefficients}
        \PY{n}{k}\PY{p}{[}\PY{n}{i}\PY{p}{]} \PY{o}{=} \PY{n}{dt}\PY{o}{*}\PY{n}{rhs\PYZus{}dict}\PY{p}{[}\PY{n}{rhs\PYZus{}key}\PY{p}{]}\PY{p}{(}\PY{n}{yhat}\PY{p}{,} \PY{n}{tn} \PY{o}{+} \PY{n}{Butcher}\PY{p}{[}\PY{n}{i}\PY{p}{]}\PY{p}{[}\PY{l+m+mi}{0}\PY{p}{]}\PY{o}{*}\PY{n}{dt}\PY{p}{)}
        \PY{c+c1}{\PYZsh{} Update the solution at the next iteration ynp1 using Butcher table coefficients}
        \PY{n}{ynp1} \PY{o}{+}\PY{o}{=} \PY{n}{Butcher}\PY{p}{[}\PY{n}{L}\PY{p}{]}\PY{p}{[}\PY{n}{i}\PY{o}{+}\PY{l+m+mi}{1}\PY{p}{]}\PY{o}{*}\PY{n}{k}\PY{p}{[}\PY{n}{i}\PY{p}{]}
    \PY{c+c1}{\PYZsh{} Finish determining the solution for the next iteration}
    \PY{n}{ynp1} \PY{o}{+}\PY{o}{=} \PY{n}{yn}

    \PY{c+c1}{\PYZsh{} Determine the order of the RK method}
    \PY{n}{order} \PY{o}{=} \PY{n}{Butcher\PYZus{}dict}\PY{p}{[}\PY{n}{Butcher\PYZus{}key}\PY{p}{]}\PY{p}{[}\PY{l+m+mi}{1}\PY{p}{]}\PY{o}{+}\PY{l+m+mi}{2}
    \PY{c+c1}{\PYZsh{} Produces Taylor series of exact solution at t=tn about t = 0 with the specified order}
    \PY{n}{exact\PYZus{}series} \PY{o}{=} \PY{n}{sp}\PY{o}{.}\PY{n}{series}\PY{p}{(}\PY{n}{exact}\PY{o}{.}\PY{n}{subs}\PY{p}{(}\PY{n}{t}\PY{p}{,} \PY{n}{dt}\PY{p}{)}\PY{p}{,}\PY{n}{dt}\PY{p}{,} \PY{l+m+mi}{0}\PY{p}{,} \PY{n}{order}\PY{p}{)}
    \PY{n}{num\PYZus{}series} \PY{o}{=} \PY{n}{sp}\PY{o}{.}\PY{n}{series}\PY{p}{(}\PY{n}{ynp1}\PY{p}{,} \PY{n}{dt}\PY{p}{,} \PY{l+m+mi}{0}\PY{p}{,} \PY{n}{order}\PY{p}{)}
    \PY{n}{diff} \PY{o}{=} \PY{n}{exact\PYZus{}series}\PY{o}{\PYZhy{}}\PY{n}{num\PYZus{}series}
    \PY{k}{return} \PY{n}{diff}
\end{Verbatim}
\end{tcolorbox}

    \hypertarget{step-2.c-validating-rk-methods-against-odes-back-to-top}{%
\subsection{\texorpdfstring{Step 2.c: Validating RK Methods against ODEs
{[}Back to
\hyperref[toc]{top}{]}}{Step 2.c: Validating RK Methods against ODEs {[}Back to {]}}}\label{step-2.c-validating-rk-methods-against-odes-back-to-top}}

\[\label{rkvalid}\]

The following makes use of the \texttt{Validate()} function above to
demonstrate that each method within the Butcher table dictionary
converges to the expected order for the given right-hand side
expression.

    \begin{tcolorbox}[breakable, size=fbox, boxrule=1pt, pad at break*=1mm,colback=cellbackground, colframe=cellborder]
\prompt{In}{incolor}{4}{\boxspacing}
\begin{Verbatim}[commandchars=\\\{\}]
\PY{n}{t}\PY{p}{,} \PY{n}{dt} \PY{o}{=} \PY{n}{sp}\PY{o}{.}\PY{n}{symbols}\PY{p}{(}\PY{l+s+s1}{\PYZsq{}}\PY{l+s+s1}{t dt}\PY{l+s+s1}{\PYZsq{}}\PY{p}{)}
\PY{c+c1}{\PYZsh{} Set initial conditions}
\PY{n}{t0} \PY{o}{=} \PY{l+m+mi}{0}
\PY{n}{y0} \PY{o}{=} \PY{l+m+mi}{1}
\PY{c+c1}{\PYZsh{} Set RHS of ODE}
\PY{n}{function} \PY{o}{=} \PY{l+s+s1}{\PYZsq{}}\PY{l+s+s1}{ypt}\PY{l+s+s1}{\PYZsq{}}\PY{c+c1}{\PYZsh{} This can be changed, just be careful that the initial conditions are satisfied}
\PY{k}{for} \PY{n}{key}\PY{p}{,}\PY{n}{value} \PY{o+ow}{in} \PY{n}{Butcher\PYZus{}dict}\PY{o}{.}\PY{n}{items}\PY{p}{(}\PY{p}{)}\PY{p}{:}
    \PY{k}{if} \PY{n}{key} \PY{o+ow}{not} \PY{o+ow}{in} \PY{p}{\PYZob{}}\PY{l+s+s2}{\PYZdq{}}\PY{l+s+s2}{AHE}\PY{l+s+s2}{\PYZdq{}}\PY{p}{,} \PY{l+s+s2}{\PYZdq{}}\PY{l+s+s2}{ABS}\PY{l+s+s2}{\PYZdq{}}\PY{p}{,} \PY{l+s+s2}{\PYZdq{}}\PY{l+s+s2}{ARKF}\PY{l+s+s2}{\PYZdq{}}\PY{p}{,} \PY{l+s+s2}{\PYZdq{}}\PY{l+s+s2}{ACK}\PY{l+s+s2}{\PYZdq{}}\PY{p}{,} \PY{l+s+s2}{\PYZdq{}}\PY{l+s+s2}{ADP5}\PY{l+s+s2}{\PYZdq{}}\PY{p}{,} \PY{l+s+s2}{\PYZdq{}}\PY{l+s+s2}{ADP8}\PY{l+s+s2}{\PYZdq{}}\PY{p}{,} \PY{l+s+s2}{\PYZdq{}}\PY{l+s+s2}{AB}\PY{l+s+s2}{\PYZdq{}}\PY{p}{\PYZcb{}}\PY{p}{:}
        \PY{n+nb}{print}\PY{p}{(}\PY{l+s+s2}{\PYZdq{}}\PY{l+s+s2}{RK method: }\PY{l+s+se}{\PYZbs{}\PYZdq{}}\PY{l+s+s2}{\PYZdq{}}\PY{o}{+}\PY{n+nb}{str}\PY{p}{(}\PY{n}{key}\PY{p}{)}\PY{o}{+}\PY{l+s+s2}{\PYZdq{}}\PY{l+s+se}{\PYZbs{}\PYZdq{}}\PY{l+s+s2}{.}\PY{l+s+s2}{\PYZdq{}}\PY{p}{)}
        \PY{n}{y} \PY{o}{=} \PY{n}{sp}\PY{o}{.}\PY{n}{Function}\PY{p}{(}\PY{l+s+s1}{\PYZsq{}}\PY{l+s+s1}{y}\PY{l+s+s1}{\PYZsq{}}\PY{p}{)}
        \PY{n+nb}{print}\PY{p}{(}\PY{l+s+s2}{\PYZdq{}}\PY{l+s+s2}{ When solving y}\PY{l+s+s2}{\PYZsq{}}\PY{l+s+s2}{(t) = }\PY{l+s+s2}{\PYZdq{}}\PY{o}{+}\PY{n+nb}{str}\PY{p}{(}\PY{n}{rhs\PYZus{}dict}\PY{p}{[}\PY{n}{function}\PY{p}{]}\PY{p}{(}\PY{n}{y}\PY{p}{(}\PY{n}{t}\PY{p}{)}\PY{p}{,}\PY{n}{t}\PY{p}{)}\PY{p}{)}\PY{o}{+}\PY{l+s+s2}{\PYZdq{}}\PY{l+s+s2}{, y(}\PY{l+s+s2}{\PYZdq{}}\PY{o}{+}\PY{n+nb}{str}\PY{p}{(}\PY{n}{t0}\PY{p}{)}\PY{o}{+}\PY{l+s+s2}{\PYZdq{}}\PY{l+s+s2}{)=}\PY{l+s+s2}{\PYZdq{}}\PY{o}{+}\PY{n+nb}{str}\PY{p}{(}\PY{n}{y0}\PY{p}{)}\PY{o}{+}\PY{l+s+s2}{\PYZdq{}}\PY{l+s+s2}{,}\PY{l+s+s2}{\PYZdq{}}\PY{p}{)}
        \PY{n}{local\PYZus{}truncation\PYZus{}order} \PY{o}{=} \PY{n+nb}{list}\PY{p}{(}\PY{n}{value}\PY{p}{)}\PY{p}{[}\PY{l+m+mi}{1}\PY{p}{]}\PY{o}{+}\PY{l+m+mi}{1}
        \PY{n+nb}{print}\PY{p}{(}\PY{l+s+s2}{\PYZdq{}}\PY{l+s+s2}{ the first nonzero term should have local truncation error proportional to O(dt\PYZca{}}\PY{l+s+s2}{\PYZdq{}}\PY{o}{+}\PY{n+nb}{str}\PY{p}{(}\PY{n}{local\PYZus{}truncation\PYZus{}order}\PY{p}{)}\PY{o}{+}\PY{l+s+s2}{\PYZdq{}}\PY{l+s+s2}{) or a higher power of dt.}\PY{l+s+s2}{\PYZdq{}}\PY{p}{)}
        \PY{n+nb}{print}\PY{p}{(}\PY{l+s+s2}{\PYZdq{}}\PY{l+s+s2}{Subtracting the numerical result from the exact Taylor expansion, we find a local truncation error of:}\PY{l+s+s2}{\PYZdq{}}\PY{p}{)}
        \PY{n}{sp}\PY{o}{.}\PY{n}{pretty\PYZus{}print}\PY{p}{(}\PY{n}{Validate}\PY{p}{(}\PY{n}{key}\PY{p}{,} \PY{n}{y0}\PY{p}{,} \PY{n}{t0}\PY{p}{,} \PY{n}{function}\PY{p}{)}\PY{p}{)}
    \PY{c+c1}{\PYZsh{}     print(\PYZdq{}\PYZbs{}n\PYZdq{})}
        \PY{n+nb}{print}\PY{p}{(}\PY{l+s+s2}{\PYZdq{}}\PY{l+s+s2}{ (Coefficients of order 1e\PYZhy{}15 or less may generally be ignored, as these are at roundoff error.)}\PY{l+s+se}{\PYZbs{}n}\PY{l+s+s2}{\PYZdq{}}\PY{p}{)}
        \PY{k}{if} \PY{n}{key} \PY{o}{==} \PY{l+s+s2}{\PYZdq{}}\PY{l+s+s2}{DP8}\PY{l+s+s2}{\PYZdq{}}\PY{p}{:}
            \PY{k}{break} \PY{c+c1}{\PYZsh{} Keep this code from trying to validate Adaptive and Adams\PYZhy{}Bashforth methods}
            \PY{c+c1}{\PYZsh{} They need to be read differently.}
\end{Verbatim}
\end{tcolorbox}

    \begin{Verbatim}[commandchars=\\\{\}]
RK method: "Euler".
 When solving y'(t) = t + y(t), y(0)=1,
 the first nonzero term should have local truncation error proportional to
O(dt\^{}2) or a higher power of dt.
Subtracting the numerical result from the exact Taylor expansion, we find a
local truncation error of:
  2    ⎛  3⎞
dt  + O⎝dt ⎠
 (Coefficients of order 1e-15 or less may generally be ignored, as these are at
roundoff error.)

RK method: "RK2 Heun".
 When solving y'(t) = t + y(t), y(0)=1,
 the first nonzero term should have local truncation error proportional to
O(dt\^{}3) or a higher power of dt.
Subtracting the numerical result from the exact Taylor expansion, we find a
local truncation error of:
  3
dt     ⎛  4⎞
─── + O⎝dt ⎠
 3
 (Coefficients of order 1e-15 or less may generally be ignored, as these are at
roundoff error.)

RK method: "RK2 MP".
 When solving y'(t) = t + y(t), y(0)=1,
 the first nonzero term should have local truncation error proportional to
O(dt\^{}3) or a higher power of dt.
Subtracting the numerical result from the exact Taylor expansion, we find a
local truncation error of:
  3
dt     ⎛  4⎞
─── + O⎝dt ⎠
 3
 (Coefficients of order 1e-15 or less may generally be ignored, as these are at
roundoff error.)

RK method: "RK2 Ralston".
 When solving y'(t) = t + y(t), y(0)=1,
 the first nonzero term should have local truncation error proportional to
O(dt\^{}3) or a higher power of dt.
Subtracting the numerical result from the exact Taylor expansion, we find a
local truncation error of:
  3
dt     ⎛  4⎞
─── + O⎝dt ⎠
 3
 (Coefficients of order 1e-15 or less may generally be ignored, as these are at
roundoff error.)

RK method: "RK3".
 When solving y'(t) = t + y(t), y(0)=1,
 the first nonzero term should have local truncation error proportional to
O(dt\^{}4) or a higher power of dt.
Subtracting the numerical result from the exact Taylor expansion, we find a
local truncation error of:
  4
dt     ⎛  5⎞
─── + O⎝dt ⎠
 12
 (Coefficients of order 1e-15 or less may generally be ignored, as these are at
roundoff error.)

RK method: "RK3 Heun".
 When solving y'(t) = t + y(t), y(0)=1,
 the first nonzero term should have local truncation error proportional to
O(dt\^{}4) or a higher power of dt.
Subtracting the numerical result from the exact Taylor expansion, we find a
local truncation error of:
  4
dt     ⎛  5⎞
─── + O⎝dt ⎠
 12
 (Coefficients of order 1e-15 or less may generally be ignored, as these are at
roundoff error.)

RK method: "RK3 Ralston".
 When solving y'(t) = t + y(t), y(0)=1,
 the first nonzero term should have local truncation error proportional to
O(dt\^{}4) or a higher power of dt.
Subtracting the numerical result from the exact Taylor expansion, we find a
local truncation error of:
  4
dt     ⎛  5⎞
─── + O⎝dt ⎠
 12
 (Coefficients of order 1e-15 or less may generally be ignored, as these are at
roundoff error.)

RK method: "SSPRK3".
 When solving y'(t) = t + y(t), y(0)=1,
 the first nonzero term should have local truncation error proportional to
O(dt\^{}4) or a higher power of dt.
Subtracting the numerical result from the exact Taylor expansion, we find a
local truncation error of:
  4
dt     ⎛  5⎞
─── + O⎝dt ⎠
 12
 (Coefficients of order 1e-15 or less may generally be ignored, as these are at
roundoff error.)

RK method: "RK4".
 When solving y'(t) = t + y(t), y(0)=1,
 the first nonzero term should have local truncation error proportional to
O(dt\^{}5) or a higher power of dt.
Subtracting the numerical result from the exact Taylor expansion, we find a
local truncation error of:
  5
dt     ⎛  6⎞
─── + O⎝dt ⎠
 60
 (Coefficients of order 1e-15 or less may generally be ignored, as these are at
roundoff error.)

RK method: "DP5".
 When solving y'(t) = t + y(t), y(0)=1,
 the first nonzero term should have local truncation error proportional to
O(dt\^{}6) or a higher power of dt.
Subtracting the numerical result from the exact Taylor expansion, we find a
local truncation error of:
    6
  dt      ⎛  7⎞
- ──── + O⎝dt ⎠
  1800
 (Coefficients of order 1e-15 or less may generally be ignored, as these are at
roundoff error.)

RK method: "DP5alt".
 When solving y'(t) = t + y(t), y(0)=1,
 the first nonzero term should have local truncation error proportional to
O(dt\^{}6) or a higher power of dt.
Subtracting the numerical result from the exact Taylor expansion, we find a
local truncation error of:
     6
13⋅dt     ⎛  7⎞
────── + O⎝dt ⎠
231000
 (Coefficients of order 1e-15 or less may generally be ignored, as these are at
roundoff error.)

RK method: "CK5".
 When solving y'(t) = t + y(t), y(0)=1,
 the first nonzero term should have local truncation error proportional to
O(dt\^{}6) or a higher power of dt.
Subtracting the numerical result from the exact Taylor expansion, we find a
local truncation error of:
  6
dt      ⎛  7⎞
──── + O⎝dt ⎠
3600
 (Coefficients of order 1e-15 or less may generally be ignored, as these are at
roundoff error.)

RK method: "DP6".
 When solving y'(t) = t + y(t), y(0)=1,
 the first nonzero term should have local truncation error proportional to
O(dt\^{}7) or a higher power of dt.
Subtracting the numerical result from the exact Taylor expansion, we find a
local truncation error of:
 ⎛  8⎞
O⎝dt ⎠
 (Coefficients of order 1e-15 or less may generally be ignored, as these are at
roundoff error.)

RK method: "L6".
 When solving y'(t) = t + y(t), y(0)=1,
 the first nonzero term should have local truncation error proportional to
O(dt\^{}7) or a higher power of dt.
Subtracting the numerical result from the exact Taylor expansion, we find a
local truncation error of:
                              5                               6
- 1.0587911840678754238e-22⋅dt  - 2.6469779601696885596e-23⋅dt  + 0.0013227513

               7    ⎛  8⎞
227513227513⋅dt  + O⎝dt ⎠
 (Coefficients of order 1e-15 or less may generally be ignored, as these are at
roundoff error.)

RK method: "DP8".
 When solving y'(t) = t + y(t), y(0)=1,
 the first nonzero term should have local truncation error proportional to
O(dt\^{}9) or a higher power of dt.
Subtracting the numerical result from the exact Taylor expansion, we find a
local truncation error of:
                                                           2
3.6854403535034607753e-18⋅dt + 5.8394451383711465375e-18⋅dt  + 3.7764963953332

             3                              4                               5
980617e-18⋅dt  + 9.542884942003761195e-19⋅dt  + 1.2718729098615353529e-19⋅dt

                              6                               7
+ 3.9082629581905451582e-20⋅dt  + 4.8075737201581968464e-21⋅dt  + 5.1688448526

                8                              9    ⎛  10⎞
907316834e-22⋅dt  + 7.2078645877627939543e-9⋅dt  + O⎝dt  ⎠
 (Coefficients of order 1e-15 or less may generally be ignored, as these are at
roundoff error.)

    \end{Verbatim}

    \hypertarget{step-2.d-validating-inherently-adaptive-rk-methods-against-odes-back-to-top}{%
\subsection{\texorpdfstring{Step 2.d: Validating Inherently Adaptive RK
Methods against ODEs {[}Back to
\hyperref[toc]{top}{]}}{Step 2.d: Validating Inherently Adaptive RK Methods against ODEs {[}Back to {]}}}\label{step-2.d-validating-inherently-adaptive-rk-methods-against-odes-back-to-top}}

\[\label{arkvalid}\]

    The following code validates inherently adaptive RK methods. Each method
has two validation checks as each method has, individually, two
different methods inside of it. One should have the first error term's
order be equal to that of the method's order, and the other check should
have the firs terror term's order be the method's order plus one.

    \begin{tcolorbox}[breakable, size=fbox, boxrule=1pt, pad at break*=1mm,colback=cellbackground, colframe=cellborder]
\prompt{In}{incolor}{5}{\boxspacing}
\begin{Verbatim}[commandchars=\\\{\}]
\PY{k}{def} \PY{n+nf}{ValidateARK1}\PY{p}{(}\PY{n}{Butcher\PYZus{}key}\PY{p}{,} \PY{n}{yn}\PY{p}{,} \PY{n}{tn}\PY{p}{,} \PY{n}{rhs\PYZus{}key}\PY{p}{)}\PY{p}{:}
    \PY{c+c1}{\PYZsh{} 1. First we solve the ODE exactly}
    \PY{n}{y} \PY{o}{=} \PY{n}{sp}\PY{o}{.}\PY{n}{Function}\PY{p}{(}\PY{l+s+s1}{\PYZsq{}}\PY{l+s+s1}{y}\PY{l+s+s1}{\PYZsq{}}\PY{p}{)}
    \PY{n}{sol} \PY{o}{=} \PY{n}{sp}\PY{o}{.}\PY{n}{dsolve}\PY{p}{(}\PY{n}{sp}\PY{o}{.}\PY{n}{Eq}\PY{p}{(}\PY{n}{y}\PY{p}{(}\PY{n}{t}\PY{p}{)}\PY{o}{.}\PY{n}{diff}\PY{p}{(}\PY{n}{t}\PY{p}{)}\PY{p}{,} \PY{n}{rhs\PYZus{}dict}\PY{p}{[}\PY{n}{rhs\PYZus{}key}\PY{p}{]}\PY{p}{(}\PY{n}{y}\PY{p}{(}\PY{n}{t}\PY{p}{)}\PY{p}{,} \PY{n}{t}\PY{p}{)}\PY{p}{)}\PY{p}{,} \PY{n}{y}\PY{p}{(}\PY{n}{t}\PY{p}{)}\PY{p}{)}\PY{o}{.}\PY{n}{rhs}
    \PY{n}{constants} \PY{o}{=} \PY{n}{sp}\PY{o}{.}\PY{n}{solve}\PY{p}{(}\PY{p}{[}\PY{n}{sol}\PY{o}{.}\PY{n}{subs}\PY{p}{(}\PY{n}{t}\PY{p}{,}\PY{n}{tn}\PY{p}{)}\PY{o}{\PYZhy{}}\PY{n}{yn}\PY{p}{]}\PY{p}{)}
    \PY{n}{exact} \PY{o}{=} \PY{n}{sol}\PY{o}{.}\PY{n}{subs}\PY{p}{(}\PY{n}{constants}\PY{p}{)}

    \PY{c+c1}{\PYZsh{} 2. Now we solve the ODE numerically using specified Butcher table}

    \PY{c+c1}{\PYZsh{} Access the requested Butcher table}
    \PY{n}{Butcher} \PY{o}{=} \PY{n}{Butcher\PYZus{}dict}\PY{p}{[}\PY{n}{Butcher\PYZus{}key}\PY{p}{]}\PY{p}{[}\PY{l+m+mi}{0}\PY{p}{]}
    \PY{c+c1}{\PYZsh{} Determine number of predictor\PYZhy{}corrector steps}
    \PY{n}{L} \PY{o}{=} \PY{n+nb}{len}\PY{p}{(}\PY{n}{Butcher}\PY{p}{)}\PY{o}{\PYZhy{}}\PY{l+m+mi}{1}
    \PY{c+c1}{\PYZsh{} Set a temporary array for update values}
    \PY{n}{k} \PY{o}{=} \PY{n}{np}\PY{o}{.}\PY{n}{zeros}\PY{p}{(}\PY{n}{L}\PY{p}{,} \PY{n}{dtype}\PY{o}{=}\PY{n+nb}{object}\PY{p}{)}
    \PY{c+c1}{\PYZsh{} Initialize intermediate variable}
    \PY{n}{yhat} \PY{o}{=} \PY{l+m+mi}{0}
    \PY{c+c1}{\PYZsh{} Initialize the updated solution}
    \PY{n}{ynp1} \PY{o}{=} \PY{l+m+mi}{0}
    \PY{k}{for} \PY{n}{i} \PY{o+ow}{in} \PY{n+nb}{range}\PY{p}{(}\PY{n}{L}\PY{o}{\PYZhy{}}\PY{l+m+mi}{1}\PY{p}{)}\PY{p}{:}
        \PY{c+c1}{\PYZsh{}Initialize and approximate update for solution}
        \PY{n}{yhat} \PY{o}{=} \PY{n}{yn}
        \PY{k}{for} \PY{n}{j} \PY{o+ow}{in} \PY{n+nb}{range}\PY{p}{(}\PY{n}{i}\PY{p}{)}\PY{p}{:}
            \PY{c+c1}{\PYZsh{} Update yhat for solution using a\PYZus{}ij Butcher table coefficients}
            \PY{n}{yhat} \PY{o}{+}\PY{o}{=} \PY{n}{Butcher}\PY{p}{[}\PY{n}{i}\PY{p}{]}\PY{p}{[}\PY{n}{j}\PY{o}{+}\PY{l+m+mi}{1}\PY{p}{]}\PY{o}{*}\PY{n}{k}\PY{p}{[}\PY{n}{j}\PY{p}{]}
            \PY{k}{if} \PY{n}{Butcher\PYZus{}key} \PY{o}{==} \PY{l+s+s2}{\PYZdq{}}\PY{l+s+s2}{DP8}\PY{l+s+s2}{\PYZdq{}} \PY{o+ow}{or} \PY{n}{Butcher\PYZus{}key} \PY{o}{==} \PY{l+s+s2}{\PYZdq{}}\PY{l+s+s2}{L6}\PY{l+s+s2}{\PYZdq{}}\PY{p}{:}
                \PY{n}{yhat} \PY{o}{=} \PY{l+m+mf}{1.0}\PY{o}{*}\PY{n}{sp}\PY{o}{.}\PY{n}{N}\PY{p}{(}\PY{n}{yhat}\PY{p}{,}\PY{l+m+mi}{20}\PY{p}{)} \PY{c+c1}{\PYZsh{} Otherwise the adding of fractions kills performance.}
        \PY{c+c1}{\PYZsh{} Determine the next corrector variable k\PYZus{}i using c\PYZus{}i Butcher table coefficients}
        \PY{n}{k}\PY{p}{[}\PY{n}{i}\PY{p}{]} \PY{o}{=} \PY{n}{dt}\PY{o}{*}\PY{n}{rhs\PYZus{}dict}\PY{p}{[}\PY{n}{rhs\PYZus{}key}\PY{p}{]}\PY{p}{(}\PY{n}{yhat}\PY{p}{,} \PY{n}{tn} \PY{o}{+} \PY{n}{Butcher}\PY{p}{[}\PY{n}{i}\PY{p}{]}\PY{p}{[}\PY{l+m+mi}{0}\PY{p}{]}\PY{o}{*}\PY{n}{dt}\PY{p}{)}
        \PY{c+c1}{\PYZsh{} Update the solution at the next iteration ynp1 using Butcher table coefficients}
        \PY{n}{ynp1} \PY{o}{+}\PY{o}{=} \PY{n}{Butcher}\PY{p}{[}\PY{n}{L}\PY{p}{]}\PY{p}{[}\PY{n}{i}\PY{o}{+}\PY{l+m+mi}{1}\PY{p}{]}\PY{o}{*}\PY{n}{k}\PY{p}{[}\PY{n}{i}\PY{p}{]}
    \PY{c+c1}{\PYZsh{} Finish determining the solution for the next iteration}
    \PY{n}{ynp1} \PY{o}{+}\PY{o}{=} \PY{n}{yn}

    \PY{c+c1}{\PYZsh{} Determine the order of the RK method}
    \PY{n}{order} \PY{o}{=} \PY{n}{Butcher\PYZus{}dict}\PY{p}{[}\PY{n}{Butcher\PYZus{}key}\PY{p}{]}\PY{p}{[}\PY{l+m+mi}{1}\PY{p}{]}\PY{o}{+}\PY{l+m+mi}{2}
    \PY{c+c1}{\PYZsh{} Produces Taylor series of exact solution at t=tn about t = 0 with the specified order}
    \PY{n}{exact\PYZus{}series} \PY{o}{=} \PY{n}{sp}\PY{o}{.}\PY{n}{series}\PY{p}{(}\PY{n}{exact}\PY{o}{.}\PY{n}{subs}\PY{p}{(}\PY{n}{t}\PY{p}{,} \PY{n}{dt}\PY{p}{)}\PY{p}{,}\PY{n}{dt}\PY{p}{,} \PY{l+m+mi}{0}\PY{p}{,} \PY{n}{order}\PY{p}{)}
    \PY{n}{num\PYZus{}series} \PY{o}{=} \PY{n}{sp}\PY{o}{.}\PY{n}{series}\PY{p}{(}\PY{n}{ynp1}\PY{p}{,} \PY{n}{dt}\PY{p}{,} \PY{l+m+mi}{0}\PY{p}{,} \PY{n}{order}\PY{p}{)}
    \PY{n}{diff} \PY{o}{=} \PY{n}{exact\PYZus{}series}\PY{o}{\PYZhy{}}\PY{n}{num\PYZus{}series}
    \PY{k}{return} \PY{n}{diff}

\PY{k}{def} \PY{n+nf}{ValidateARK2}\PY{p}{(}\PY{n}{Butcher\PYZus{}key}\PY{p}{,} \PY{n}{yn}\PY{p}{,} \PY{n}{tn}\PY{p}{,} \PY{n}{rhs\PYZus{}key}\PY{p}{)}\PY{p}{:}
    \PY{c+c1}{\PYZsh{} 1. First we solve the ODE exactly}
    \PY{n}{y} \PY{o}{=} \PY{n}{sp}\PY{o}{.}\PY{n}{Function}\PY{p}{(}\PY{l+s+s1}{\PYZsq{}}\PY{l+s+s1}{y}\PY{l+s+s1}{\PYZsq{}}\PY{p}{)}
    \PY{n}{sol} \PY{o}{=} \PY{n}{sp}\PY{o}{.}\PY{n}{dsolve}\PY{p}{(}\PY{n}{sp}\PY{o}{.}\PY{n}{Eq}\PY{p}{(}\PY{n}{y}\PY{p}{(}\PY{n}{t}\PY{p}{)}\PY{o}{.}\PY{n}{diff}\PY{p}{(}\PY{n}{t}\PY{p}{)}\PY{p}{,} \PY{n}{rhs\PYZus{}dict}\PY{p}{[}\PY{n}{rhs\PYZus{}key}\PY{p}{]}\PY{p}{(}\PY{n}{y}\PY{p}{(}\PY{n}{t}\PY{p}{)}\PY{p}{,} \PY{n}{t}\PY{p}{)}\PY{p}{)}\PY{p}{,} \PY{n}{y}\PY{p}{(}\PY{n}{t}\PY{p}{)}\PY{p}{)}\PY{o}{.}\PY{n}{rhs}
    \PY{n}{constants} \PY{o}{=} \PY{n}{sp}\PY{o}{.}\PY{n}{solve}\PY{p}{(}\PY{p}{[}\PY{n}{sol}\PY{o}{.}\PY{n}{subs}\PY{p}{(}\PY{n}{t}\PY{p}{,}\PY{n}{tn}\PY{p}{)}\PY{o}{\PYZhy{}}\PY{n}{yn}\PY{p}{]}\PY{p}{)}
    \PY{n}{exact} \PY{o}{=} \PY{n}{sol}\PY{o}{.}\PY{n}{subs}\PY{p}{(}\PY{n}{constants}\PY{p}{)}

    \PY{c+c1}{\PYZsh{} 2. Now we solve the ODE numerically using specified Butcher table}

    \PY{c+c1}{\PYZsh{} Access the requested Butcher table}
    \PY{n}{Butcher} \PY{o}{=} \PY{n}{Butcher\PYZus{}dict}\PY{p}{[}\PY{n}{Butcher\PYZus{}key}\PY{p}{]}\PY{p}{[}\PY{l+m+mi}{0}\PY{p}{]}
    \PY{c+c1}{\PYZsh{} Determine number of predictor\PYZhy{}corrector steps}
    \PY{n}{L} \PY{o}{=} \PY{n+nb}{len}\PY{p}{(}\PY{n}{Butcher}\PY{p}{)}\PY{o}{\PYZhy{}}\PY{l+m+mi}{1}
    \PY{c+c1}{\PYZsh{} Set a temporary array for update values}
    \PY{n}{k} \PY{o}{=} \PY{n}{np}\PY{o}{.}\PY{n}{zeros}\PY{p}{(}\PY{n}{L}\PY{p}{,} \PY{n}{dtype}\PY{o}{=}\PY{n+nb}{object}\PY{p}{)}
    \PY{c+c1}{\PYZsh{} Initialize intermediate variable}
    \PY{n}{yhat} \PY{o}{=} \PY{l+m+mi}{0}
    \PY{c+c1}{\PYZsh{} Initialize the updated solution}
    \PY{n}{ynp1} \PY{o}{=} \PY{l+m+mi}{0}
    \PY{k}{for} \PY{n}{i} \PY{o+ow}{in} \PY{n+nb}{range}\PY{p}{(}\PY{n}{L}\PY{o}{\PYZhy{}}\PY{l+m+mi}{1}\PY{p}{)}\PY{p}{:}
        \PY{c+c1}{\PYZsh{}Initialize and approximate update for solution}
        \PY{n}{yhat} \PY{o}{=} \PY{n}{yn}
        \PY{k}{for} \PY{n}{j} \PY{o+ow}{in} \PY{n+nb}{range}\PY{p}{(}\PY{n}{i}\PY{p}{)}\PY{p}{:}
            \PY{c+c1}{\PYZsh{} Update yhat for solution using a\PYZus{}ij Butcher table coefficients}
            \PY{n}{yhat} \PY{o}{+}\PY{o}{=} \PY{n}{Butcher}\PY{p}{[}\PY{n}{i}\PY{p}{]}\PY{p}{[}\PY{n}{j}\PY{o}{+}\PY{l+m+mi}{1}\PY{p}{]}\PY{o}{*}\PY{n}{k}\PY{p}{[}\PY{n}{j}\PY{p}{]}
            \PY{k}{if} \PY{n}{Butcher\PYZus{}key} \PY{o}{==} \PY{l+s+s2}{\PYZdq{}}\PY{l+s+s2}{DP8}\PY{l+s+s2}{\PYZdq{}} \PY{o+ow}{or} \PY{n}{Butcher\PYZus{}key} \PY{o}{==} \PY{l+s+s2}{\PYZdq{}}\PY{l+s+s2}{L6}\PY{l+s+s2}{\PYZdq{}}\PY{p}{:}
                \PY{n}{yhat} \PY{o}{=} \PY{l+m+mf}{1.0}\PY{o}{*}\PY{n}{sp}\PY{o}{.}\PY{n}{N}\PY{p}{(}\PY{n}{yhat}\PY{p}{,}\PY{l+m+mi}{20}\PY{p}{)} \PY{c+c1}{\PYZsh{} Otherwise the adding of fractions kills performance.}
        \PY{c+c1}{\PYZsh{} Determine the next corrector variable k\PYZus{}i using c\PYZus{}i Butcher table coefficients}
        \PY{n}{k}\PY{p}{[}\PY{n}{i}\PY{p}{]} \PY{o}{=} \PY{n}{dt}\PY{o}{*}\PY{n}{rhs\PYZus{}dict}\PY{p}{[}\PY{n}{rhs\PYZus{}key}\PY{p}{]}\PY{p}{(}\PY{n}{yhat}\PY{p}{,} \PY{n}{tn} \PY{o}{+} \PY{n}{Butcher}\PY{p}{[}\PY{n}{i}\PY{p}{]}\PY{p}{[}\PY{l+m+mi}{0}\PY{p}{]}\PY{o}{*}\PY{n}{dt}\PY{p}{)}
        \PY{c+c1}{\PYZsh{} Update the solution at the next iteration ynp1 using Butcher table coefficients}
        \PY{n}{ynp1} \PY{o}{+}\PY{o}{=} \PY{n}{Butcher}\PY{p}{[}\PY{n}{L}\PY{o}{\PYZhy{}}\PY{l+m+mi}{1}\PY{p}{]}\PY{p}{[}\PY{n}{i}\PY{o}{+}\PY{l+m+mi}{1}\PY{p}{]}\PY{o}{*}\PY{n}{k}\PY{p}{[}\PY{n}{i}\PY{p}{]}
    \PY{c+c1}{\PYZsh{} Finish determining the solution for the next iteration}
    \PY{n}{ynp1} \PY{o}{+}\PY{o}{=} \PY{n}{yn}

    \PY{c+c1}{\PYZsh{} Determine the order of the RK method}
    \PY{n}{order} \PY{o}{=} \PY{n}{Butcher\PYZus{}dict}\PY{p}{[}\PY{n}{Butcher\PYZus{}key}\PY{p}{]}\PY{p}{[}\PY{l+m+mi}{1}\PY{p}{]}\PY{o}{+}\PY{l+m+mi}{2}
    \PY{c+c1}{\PYZsh{} Produces Taylor series of exact solution at t=tn about t = 0 with the specified order}
    \PY{n}{exact\PYZus{}series} \PY{o}{=} \PY{n}{sp}\PY{o}{.}\PY{n}{series}\PY{p}{(}\PY{n}{exact}\PY{o}{.}\PY{n}{subs}\PY{p}{(}\PY{n}{t}\PY{p}{,} \PY{n}{dt}\PY{p}{)}\PY{p}{,}\PY{n}{dt}\PY{p}{,} \PY{l+m+mi}{0}\PY{p}{,} \PY{n}{order}\PY{p}{)}
    \PY{n}{num\PYZus{}series} \PY{o}{=} \PY{n}{sp}\PY{o}{.}\PY{n}{series}\PY{p}{(}\PY{n}{ynp1}\PY{p}{,} \PY{n}{dt}\PY{p}{,} \PY{l+m+mi}{0}\PY{p}{,} \PY{n}{order}\PY{p}{)}
    \PY{n}{diff} \PY{o}{=} \PY{n}{exact\PYZus{}series}\PY{o}{\PYZhy{}}\PY{n}{num\PYZus{}series}
    \PY{k}{return} \PY{n}{diff}
\end{Verbatim}
\end{tcolorbox}

    \begin{tcolorbox}[breakable, size=fbox, boxrule=1pt, pad at break*=1mm,colback=cellbackground, colframe=cellborder]
\prompt{In}{incolor}{6}{\boxspacing}
\begin{Verbatim}[commandchars=\\\{\}]
\PY{n}{t}\PY{p}{,} \PY{n}{dt} \PY{o}{=} \PY{n}{sp}\PY{o}{.}\PY{n}{symbols}\PY{p}{(}\PY{l+s+s1}{\PYZsq{}}\PY{l+s+s1}{t dt}\PY{l+s+s1}{\PYZsq{}}\PY{p}{)}
\PY{c+c1}{\PYZsh{} Set initial conditions}
\PY{n}{t0} \PY{o}{=} \PY{l+m+mi}{0}
\PY{n}{y0} \PY{o}{=} \PY{l+m+mi}{1}
\PY{c+c1}{\PYZsh{} Set RHS of ODE}
\PY{n}{toggle} \PY{o}{=} \PY{l+m+mi}{0} \PY{c+c1}{\PYZsh{} This is a bookkeeping device for knowing when we reached the Inherently Adaptive methods}
\PY{k}{for} \PY{n}{key}\PY{p}{,}\PY{n}{value} \PY{o+ow}{in} \PY{n}{Butcher\PYZus{}dict}\PY{o}{.}\PY{n}{items}\PY{p}{(}\PY{p}{)}\PY{p}{:}
    \PY{k}{if} \PY{n}{key} \PY{o+ow}{in} \PY{p}{\PYZob{}}\PY{l+s+s2}{\PYZdq{}}\PY{l+s+s2}{AHE}\PY{l+s+s2}{\PYZdq{}}\PY{p}{,} \PY{l+s+s2}{\PYZdq{}}\PY{l+s+s2}{ABS}\PY{l+s+s2}{\PYZdq{}}\PY{p}{,} \PY{l+s+s2}{\PYZdq{}}\PY{l+s+s2}{ARKF}\PY{l+s+s2}{\PYZdq{}}\PY{p}{,} \PY{l+s+s2}{\PYZdq{}}\PY{l+s+s2}{ACK}\PY{l+s+s2}{\PYZdq{}}\PY{p}{,} \PY{l+s+s2}{\PYZdq{}}\PY{l+s+s2}{ADP5}\PY{l+s+s2}{\PYZdq{}}\PY{p}{,} \PY{l+s+s2}{\PYZdq{}}\PY{l+s+s2}{ADP8}\PY{l+s+s2}{\PYZdq{}}\PY{p}{\PYZcb{}}\PY{p}{:}
        \PY{k}{if} \PY{p}{(}\PY{n}{key} \PY{o}{==} \PY{l+s+s2}{\PYZdq{}}\PY{l+s+s2}{AHE}\PY{l+s+s2}{\PYZdq{}}\PY{p}{)}\PY{p}{:}
            \PY{n}{toggle} \PY{o}{=} \PY{l+m+mi}{1}
        \PY{k}{if} \PY{p}{(}\PY{n}{toggle} \PY{o}{==} \PY{l+m+mi}{1}\PY{p}{)}\PY{p}{:} \PY{c+c1}{\PYZsh{}only do anything once we actually hit adaptive methods.}
            \PY{n+nb}{print}\PY{p}{(}\PY{l+s+s2}{\PYZdq{}}\PY{l+s+s2}{RK method: }\PY{l+s+se}{\PYZbs{}\PYZdq{}}\PY{l+s+s2}{\PYZdq{}}\PY{o}{+}\PY{n+nb}{str}\PY{p}{(}\PY{n}{key}\PY{p}{)}\PY{o}{+}\PY{l+s+s2}{\PYZdq{}}\PY{l+s+se}{\PYZbs{}\PYZdq{}}\PY{l+s+s2}{.}\PY{l+s+s2}{\PYZdq{}}\PY{p}{)}
            \PY{n}{y} \PY{o}{=} \PY{n}{sp}\PY{o}{.}\PY{n}{Function}\PY{p}{(}\PY{l+s+s1}{\PYZsq{}}\PY{l+s+s1}{y}\PY{l+s+s1}{\PYZsq{}}\PY{p}{)}
            \PY{n+nb}{print}\PY{p}{(}\PY{l+s+s2}{\PYZdq{}}\PY{l+s+s2}{ When solving y}\PY{l+s+s2}{\PYZsq{}}\PY{l+s+s2}{(t) = }\PY{l+s+s2}{\PYZdq{}}\PY{o}{+}\PY{n+nb}{str}\PY{p}{(}\PY{n}{rhs\PYZus{}dict}\PY{p}{[}\PY{n}{function}\PY{p}{]}\PY{p}{(}\PY{n}{y}\PY{p}{(}\PY{n}{t}\PY{p}{)}\PY{p}{,}\PY{n}{t}\PY{p}{)}\PY{p}{)}\PY{o}{+}\PY{l+s+s2}{\PYZdq{}}\PY{l+s+s2}{, y(}\PY{l+s+s2}{\PYZdq{}}\PY{o}{+}\PY{n+nb}{str}\PY{p}{(}\PY{n}{t0}\PY{p}{)}\PY{o}{+}\PY{l+s+s2}{\PYZdq{}}\PY{l+s+s2}{)=}\PY{l+s+s2}{\PYZdq{}}\PY{o}{+}\PY{n+nb}{str}\PY{p}{(}\PY{n}{y0}\PY{p}{)}\PY{o}{+}\PY{l+s+s2}{\PYZdq{}}\PY{l+s+s2}{,}\PY{l+s+s2}{\PYZdq{}}\PY{p}{)}
            \PY{n}{local\PYZus{}truncation\PYZus{}order} \PY{o}{=} \PY{n+nb}{list}\PY{p}{(}\PY{n}{value}\PY{p}{)}\PY{p}{[}\PY{l+m+mi}{1}\PY{p}{]}\PY{o}{+}\PY{l+m+mi}{1}
            \PY{n+nb}{print}\PY{p}{(}\PY{l+s+s2}{\PYZdq{}}\PY{l+s+s2}{ the first calculation}\PY{l+s+s2}{\PYZsq{}}\PY{l+s+s2}{s first nonzero term should have local truncation error proportional to O(dt\PYZca{}}\PY{l+s+s2}{\PYZdq{}}\PY{o}{+}\PY{n+nb}{str}\PY{p}{(}\PY{n}{local\PYZus{}truncation\PYZus{}order}\PY{o}{\PYZhy{}}\PY{l+m+mi}{1}\PY{p}{)}\PY{o}{+}\PY{l+s+s2}{\PYZdq{}}\PY{l+s+s2}{) or a higher power of dt.}\PY{l+s+s2}{\PYZdq{}}\PY{p}{)}
            \PY{n+nb}{print}\PY{p}{(}\PY{l+s+s2}{\PYZdq{}}\PY{l+s+s2}{ the second calculation}\PY{l+s+s2}{\PYZsq{}}\PY{l+s+s2}{s first nonzero term should have local truncation error proportional to O(dt\PYZca{}}\PY{l+s+s2}{\PYZdq{}}\PY{o}{+}\PY{n+nb}{str}\PY{p}{(}\PY{n}{local\PYZus{}truncation\PYZus{}order}\PY{p}{)}\PY{o}{+}\PY{l+s+s2}{\PYZdq{}}\PY{l+s+s2}{) or a higher power of dt.}\PY{l+s+s2}{\PYZdq{}}\PY{p}{)}
            \PY{n+nb}{print}\PY{p}{(}\PY{l+s+s2}{\PYZdq{}}\PY{l+s+s2}{Subtracting the numerical result from the exact Taylor expansion, we find a local truncation error of:}\PY{l+s+s2}{\PYZdq{}}\PY{p}{)}
            \PY{k}{if} \PY{n}{key} \PY{o}{==} \PY{l+s+s2}{\PYZdq{}}\PY{l+s+s2}{ADP8}\PY{l+s+s2}{\PYZdq{}}\PY{p}{:} \PY{c+c1}{\PYZsh{}ADP8 takes up too much of the screen in analytic, we need to print as decimals.}
                \PY{n}{sp}\PY{o}{.}\PY{n}{pretty\PYZus{}print}\PY{p}{(}\PY{n}{ValidateARK1}\PY{p}{(}\PY{n}{key}\PY{p}{,} \PY{n}{y0}\PY{p}{,} \PY{n}{t0}\PY{p}{,} \PY{n}{function}\PY{p}{)}\PY{o}{.}\PY{n}{evalf}\PY{p}{(}\PY{p}{)}\PY{p}{)}
                \PY{n}{sp}\PY{o}{.}\PY{n}{pretty\PYZus{}print}\PY{p}{(}\PY{n}{ValidateARK2}\PY{p}{(}\PY{n}{key}\PY{p}{,} \PY{n}{y0}\PY{p}{,} \PY{n}{t0}\PY{p}{,} \PY{n}{function}\PY{p}{)}\PY{o}{.}\PY{n}{evalf}\PY{p}{(}\PY{p}{)}\PY{p}{)}
            \PY{k}{else}\PY{p}{:}
                \PY{n}{sp}\PY{o}{.}\PY{n}{pretty\PYZus{}print}\PY{p}{(}\PY{n}{ValidateARK1}\PY{p}{(}\PY{n}{key}\PY{p}{,} \PY{n}{y0}\PY{p}{,} \PY{n}{t0}\PY{p}{,} \PY{n}{function}\PY{p}{)}\PY{p}{)}
                \PY{n}{sp}\PY{o}{.}\PY{n}{pretty\PYZus{}print}\PY{p}{(}\PY{n}{ValidateARK2}\PY{p}{(}\PY{n}{key}\PY{p}{,} \PY{n}{y0}\PY{p}{,} \PY{n}{t0}\PY{p}{,} \PY{n}{function}\PY{p}{)}\PY{p}{)}
        \PY{c+c1}{\PYZsh{}     print(\PYZdq{}\PYZbs{}n\PYZdq{})}
            \PY{n+nb}{print}\PY{p}{(}\PY{l+s+s2}{\PYZdq{}}\PY{l+s+s2}{ (Coefficients of order 1e\PYZhy{}15 or less may generally be ignored, as these are at roundoff error.)}\PY{l+s+se}{\PYZbs{}n}\PY{l+s+s2}{\PYZdq{}}\PY{p}{)}
\end{Verbatim}
\end{tcolorbox}

    \begin{Verbatim}[commandchars=\\\{\}]
RK method: "AHE".
 When solving y'(t) = t + y(t), y(0)=1,
 the first calculation's first nonzero term should have local truncation error
proportional to O(dt\^{}2) or a higher power of dt.
 the second calculation's first nonzero term should have local truncation error
proportional to O(dt\^{}3) or a higher power of dt.
Subtracting the numerical result from the exact Taylor expansion, we find a
local truncation error of:
        3
  2   dt     ⎛  4⎞
dt  + ─── + O⎝dt ⎠
       3
  3
dt     ⎛  4⎞
─── + O⎝dt ⎠
 3
 (Coefficients of order 1e-15 or less may generally be ignored, as these are at
roundoff error.)

RK method: "ABS".
 When solving y'(t) = t + y(t), y(0)=1,
 the first calculation's first nonzero term should have local truncation error
proportional to O(dt\^{}3) or a higher power of dt.
 the second calculation's first nonzero term should have local truncation error
proportional to O(dt\^{}4) or a higher power of dt.
Subtracting the numerical result from the exact Taylor expansion, we find a
local truncation error of:
    3     4
  dt    dt     ⎛  5⎞
- ─── + ─── + O⎝dt ⎠
   24    24
  4
dt     ⎛  5⎞
─── + O⎝dt ⎠
 12
 (Coefficients of order 1e-15 or less may generally be ignored, as these are at
roundoff error.)

RK method: "ARKF".
 When solving y'(t) = t + y(t), y(0)=1,
 the first calculation's first nonzero term should have local truncation error
proportional to O(dt\^{}5) or a higher power of dt.
 the second calculation's first nonzero term should have local truncation error
proportional to O(dt\^{}6) or a higher power of dt.
Subtracting the numerical result from the exact Taylor expansion, we find a
local truncation error of:
    5     6
  dt    dt     ⎛  7⎞
- ─── + ─── + O⎝dt ⎠
  390   360
     6
17⋅dt     ⎛  7⎞
────── + O⎝dt ⎠
 9360
 (Coefficients of order 1e-15 or less may generally be ignored, as these are at
roundoff error.)

RK method: "ACK".
 When solving y'(t) = t + y(t), y(0)=1,
 the first calculation's first nonzero term should have local truncation error
proportional to O(dt\^{}5) or a higher power of dt.
 the second calculation's first nonzero term should have local truncation error
proportional to O(dt\^{}6) or a higher power of dt.
Subtracting the numerical result from the exact Taylor expansion, we find a
local truncation error of:
        5          6
  277⋅dt    4541⋅dt     ⎛  7⎞
- ─────── + ──────── + O⎝dt ⎠
   614400   7372800
  6
dt      ⎛  7⎞
──── + O⎝dt ⎠
3600
 (Coefficients of order 1e-15 or less may generally be ignored, as these are at
roundoff error.)

RK method: "ADP5".
 When solving y'(t) = t + y(t), y(0)=1,
 the first calculation's first nonzero term should have local truncation error
proportional to O(dt\^{}5) or a higher power of dt.
 the second calculation's first nonzero term should have local truncation error
proportional to O(dt\^{}6) or a higher power of dt.
Subtracting the numerical result from the exact Taylor expansion, we find a
local truncation error of:
       5        6
  97⋅dt    17⋅dt     ⎛  7⎞
- ────── + ────── + O⎝dt ⎠
  60000    180000
    6
  dt      ⎛  7⎞
- ──── + O⎝dt ⎠
  1800
 (Coefficients of order 1e-15 or less may generally be ignored, as these are at
roundoff error.)

RK method: "ADP8".
 When solving y'(t) = t + y(t), y(0)=1,
 the first calculation's first nonzero term should have local truncation error
proportional to O(dt\^{}8) or a higher power of dt.
 the second calculation's first nonzero term should have local truncation error
proportional to O(dt\^{}9) or a higher power of dt.
Subtracting the numerical result from the exact Taylor expansion, we find a
local truncation error of:
                                                  2                          3
-7.71097267488672e-19⋅dt + 2.91023006428162e-18⋅dt  + 5.29778138968287e-19⋅dt

                         4                          5
 - 1.5349356222021e-19⋅dt  + 1.65528504999405e-19⋅dt  + 4.19247608610368e-20⋅d

 6                          7                         8
t  + 6.27844437078994e-21⋅dt  - 4.85333183539141e-7⋅dt  + 3.49344710134931e-7⋅

  9    ⎛  10⎞
dt  + O⎝dt  ⎠
                                                 2                          3
3.68531467298237e-18⋅dt + 5.84980541251108e-18⋅dt  + 3.78072846284061e-18⋅dt

                         4                          5
+ 9.53606673846474e-19⋅dt  + 1.26972675407126e-19⋅dt  + 3.90778437343018e-20⋅d

 6                          7                          8
t  + 4.80748915688373e-21⋅dt  + 5.17189453562972e-22⋅dt  + 7.20786458776279e-9

   9    ⎛  10⎞
⋅dt  + O⎝dt  ⎠
 (Coefficients of order 1e-15 or less may generally be ignored, as these are at
roundoff error.)

    \end{Verbatim}

    \hypertarget{step-2.e-validating-ab-methods-against-odes-back-to-top}{%
\subsection{\texorpdfstring{Step 2.e: Validating AB Methods against ODEs
{[}Back to
\hyperref[toc]{top}{]}}{Step 2.e: Validating AB Methods against ODEs {[}Back to {]}}}\label{step-2.e-validating-ab-methods-against-odes-back-to-top}}

\[\label{abvalid}\]

    The following code validates the AB methods, all of them from order 1 to
19. Their first remaining error term should be precisely one order
higher than the method order.

Since AB methods have to validate using multiple past values, the
validation function is hard-coded in, since we need exact solutions at
those past points for this to work. Fortunately \(y'=y\) is suitable for
this purpose, as it sufficiently tests all the orders up to 19.

    \begin{tcolorbox}[breakable, size=fbox, boxrule=1pt, pad at break*=1mm,colback=cellbackground, colframe=cellborder]
\prompt{In}{incolor}{7}{\boxspacing}
\begin{Verbatim}[commandchars=\\\{\}]
\PY{c+c1}{\PYZsh{}This is taken almost directly from the butcher validation from NRPy+}

\PY{k}{def} \PY{n+nf}{ValidateAB}\PY{p}{(}\PY{n}{ABorder}\PY{p}{,} \PY{n}{yn}\PY{p}{,} \PY{n}{tn}\PY{p}{,} \PY{n}{rhs\PYZus{}key}\PY{p}{)}\PY{p}{:} \PY{c+c1}{\PYZsh{} custom function for validating AB methods.}

    \PY{n}{order} \PY{o}{=} \PY{n}{ABorder}

    \PY{c+c1}{\PYZsh{} 1. First we solve the ODE exactly}
    \PY{n}{y} \PY{o}{=} \PY{n}{sp}\PY{o}{.}\PY{n}{Function}\PY{p}{(}\PY{l+s+s1}{\PYZsq{}}\PY{l+s+s1}{y}\PY{l+s+s1}{\PYZsq{}}\PY{p}{)}
    \PY{n}{sol} \PY{o}{=} \PY{n}{sp}\PY{o}{.}\PY{n}{dsolve}\PY{p}{(}\PY{n}{sp}\PY{o}{.}\PY{n}{Eq}\PY{p}{(}\PY{n}{y}\PY{p}{(}\PY{n}{t}\PY{p}{)}\PY{o}{.}\PY{n}{diff}\PY{p}{(}\PY{n}{t}\PY{p}{)}\PY{p}{,} \PY{n}{fy}\PY{p}{(}\PY{n}{y}\PY{p}{(}\PY{n}{t}\PY{p}{)}\PY{p}{,} \PY{n}{t}\PY{p}{)}\PY{p}{)}\PY{p}{,} \PY{n}{y}\PY{p}{(}\PY{n}{t}\PY{p}{)}\PY{p}{)}\PY{o}{.}\PY{n}{rhs}
    \PY{n}{constants} \PY{o}{=} \PY{n}{sp}\PY{o}{.}\PY{n}{solve}\PY{p}{(}\PY{p}{[}\PY{n}{sol}\PY{o}{.}\PY{n}{subs}\PY{p}{(}\PY{n}{t}\PY{p}{,}\PY{n}{tn}\PY{p}{)}\PY{o}{\PYZhy{}}\PY{n}{yn}\PY{p}{]}\PY{p}{)}
    \PY{n}{exact} \PY{o}{=} \PY{n}{sol}\PY{o}{.}\PY{n}{subs}\PY{p}{(}\PY{n}{constants}\PY{p}{)}
    \PY{n}{exact\PYZus{}series} \PY{o}{=} \PY{n}{sp}\PY{o}{.}\PY{n}{series}\PY{p}{(}\PY{n}{exact}\PY{o}{.}\PY{n}{subs}\PY{p}{(}\PY{n}{t}\PY{p}{,} \PY{n}{dt}\PY{p}{)}\PY{p}{,}\PY{n}{dt}\PY{p}{,} \PY{l+m+mi}{0}\PY{p}{,} \PY{n}{order}\PY{o}{+}\PY{l+m+mi}{2}\PY{p}{)}

    \PY{c+c1}{\PYZsh{} 2. Now we solve the ODE numerically using specified Butcher table}

    \PY{c+c1}{\PYZsh{} Access the requested Butcher table}
    \PY{n}{Butcher} \PY{o}{=} \PY{n}{Butcher\PYZus{}dict}\PY{o}{.}\PY{n}{get}\PY{p}{(}\PY{l+s+s2}{\PYZdq{}}\PY{l+s+s2}{AB}\PY{l+s+s2}{\PYZdq{}}\PY{p}{)}\PY{p}{[}\PY{l+m+mi}{0}\PY{p}{]}
    \PY{c+c1}{\PYZsh{} Determine number of predictor\PYZhy{}corrector steps}
    \PY{n}{L} \PY{o}{=} \PY{n+nb}{len}\PY{p}{(}\PY{n}{Butcher}\PY{p}{)}\PY{o}{\PYZhy{}}\PY{l+m+mi}{1} \PY{c+c1}{\PYZsh{}set to \PYZhy{}2 for adaptive methods.}
    \PY{c+c1}{\PYZsh{} Set a temporary array for update values}
    \PY{n}{k} \PY{o}{=} \PY{n}{np}\PY{o}{.}\PY{n}{zeros}\PY{p}{(}\PY{n}{L}\PY{p}{,} \PY{n}{dtype}\PY{o}{=}\PY{n+nb}{object}\PY{p}{)}
    \PY{c+c1}{\PYZsh{} Initialize intermediate variable}
    \PY{n}{yhat} \PY{o}{=} \PY{l+m+mi}{1}
    \PY{c+c1}{\PYZsh{} Initialize the updated solution}
    \PY{n}{ynp1} \PY{o}{=} \PY{l+m+mi}{0}
    \PY{k}{for} \PY{n}{i} \PY{o+ow}{in} \PY{n+nb}{range}\PY{p}{(}\PY{n}{ABorder}\PY{p}{)}\PY{p}{:}
        \PY{c+c1}{\PYZsh{} Initialize and approximate update for solution}
        \PY{c+c1}{\PYZsh{} Update yhat for solution using \PYZdq{}past values\PYZdq{},}
        \PY{c+c1}{\PYZsh{} Which means evaluating the exact answer at x=(\PYZhy{}i*dt).}
        \PY{c+c1}{\PYZsh{} If the user wishes to validate another ODE, the respective section}
        \PY{c+c1}{\PYZsh{} In the below line will have to be changed.}
        \PY{n}{yhat} \PY{o}{+}\PY{o}{=} \PY{n}{dt}\PY{o}{*}\PY{n}{Butcher}\PY{p}{[}\PY{n}{ABorder}\PY{o}{\PYZhy{}}\PY{l+m+mi}{1}\PY{p}{]}\PY{p}{[}\PY{n}{i}\PY{p}{]}\PY{o}{*}\PY{n}{sp}\PY{o}{.}\PY{n}{exp}\PY{p}{(}\PY{p}{(}\PY{o}{\PYZhy{}}\PY{n}{i}\PY{p}{)}\PY{o}{*}\PY{n}{dt}\PY{p}{)}
    \PY{c+c1}{\PYZsh{} Finish determining the solution for the next iteration}
    \PY{n}{num\PYZus{}series} \PY{o}{=} \PY{n}{sp}\PY{o}{.}\PY{n}{series}\PY{p}{(}\PY{n}{yhat}\PY{p}{,} \PY{n}{dt}\PY{p}{,} \PY{l+m+mi}{0}\PY{p}{,} \PY{n}{order}\PY{o}{+}\PY{l+m+mi}{2}\PY{p}{)}

    \PY{c+c1}{\PYZsh{} Produces Taylor series of exact solution at t=tn about t = 0 with the specified order}
    \PY{n}{diff} \PY{o}{=} \PY{n}{exact\PYZus{}series}\PY{o}{\PYZhy{}}\PY{n}{num\PYZus{}series}
    \PY{k}{return} \PY{n}{diff}
\end{Verbatim}
\end{tcolorbox}

    \begin{tcolorbox}[breakable, size=fbox, boxrule=1pt, pad at break*=1mm,colback=cellbackground, colframe=cellborder]
\prompt{In}{incolor}{8}{\boxspacing}
\begin{Verbatim}[commandchars=\\\{\}]
\PY{n}{t}\PY{p}{,} \PY{n}{dt} \PY{o}{=} \PY{n}{sp}\PY{o}{.}\PY{n}{symbols}\PY{p}{(}\PY{l+s+s1}{\PYZsq{}}\PY{l+s+s1}{t dt}\PY{l+s+s1}{\PYZsq{}}\PY{p}{)}
\PY{c+c1}{\PYZsh{} Set initial conditions}
\PY{n}{t0} \PY{o}{=} \PY{l+m+mi}{0}
\PY{n}{y0} \PY{o}{=} \PY{l+m+mi}{1}

\PY{n}{i} \PY{o}{=} \PY{l+m+mi}{1}
\PY{k}{while} \PY{n}{i} \PY{o}{\PYZlt{}} \PY{l+m+mi}{20}\PY{p}{:}
    \PY{n+nb}{print}\PY{p}{(}\PY{l+s+s2}{\PYZdq{}}\PY{l+s+s2}{AB method order: }\PY{l+s+se}{\PYZbs{}\PYZdq{}}\PY{l+s+s2}{\PYZdq{}}\PY{o}{+}\PY{n+nb}{str}\PY{p}{(}\PY{n}{i}\PY{p}{)}\PY{o}{+}\PY{l+s+s2}{\PYZdq{}}\PY{l+s+se}{\PYZbs{}\PYZdq{}}\PY{l+s+s2}{.}\PY{l+s+s2}{\PYZdq{}}\PY{p}{)}
    \PY{n}{y} \PY{o}{=} \PY{n}{sp}\PY{o}{.}\PY{n}{Function}\PY{p}{(}\PY{l+s+s1}{\PYZsq{}}\PY{l+s+s1}{y}\PY{l+s+s1}{\PYZsq{}}\PY{p}{)}
    \PY{n+nb}{print}\PY{p}{(}\PY{l+s+s2}{\PYZdq{}}\PY{l+s+s2}{ When solving y}\PY{l+s+s2}{\PYZsq{}}\PY{l+s+s2}{(t) = y, y(0)=1,}\PY{l+s+s2}{\PYZdq{}}\PY{p}{)} \PY{c+c1}{\PYZsh{} make this adaptable to all possible inputs.}
    \PY{n+nb}{print}\PY{p}{(}\PY{l+s+s2}{\PYZdq{}}\PY{l+s+s2}{ the first nonzero term should have local truncation error proportional to O(dt\PYZca{}}\PY{l+s+s2}{\PYZdq{}}\PY{o}{+}\PY{n+nb}{str}\PY{p}{(}\PY{n}{i}\PY{o}{+}\PY{l+m+mi}{1}\PY{p}{)}\PY{o}{+}\PY{l+s+s2}{\PYZdq{}}\PY{l+s+s2}{) or a higher power of dt.}\PY{l+s+s2}{\PYZdq{}}\PY{p}{)}
    \PY{n+nb}{print}\PY{p}{(}\PY{l+s+s2}{\PYZdq{}}\PY{l+s+s2}{Subtracting the numerical result from the exact Taylor expansion, we find a local truncation error of:}\PY{l+s+s2}{\PYZdq{}}\PY{p}{)}
    \PY{n}{sp}\PY{o}{.}\PY{n}{pretty\PYZus{}print}\PY{p}{(}\PY{n}{ValidateAB}\PY{p}{(}\PY{n}{i}\PY{p}{,} \PY{n}{y0}\PY{p}{,} \PY{n}{t0}\PY{p}{,} \PY{n}{function}\PY{p}{)}\PY{p}{)}
\PY{c+c1}{\PYZsh{}     print(\PYZdq{}\PYZbs{}n\PYZdq{})}
    \PY{n+nb}{print}\PY{p}{(}\PY{l+s+s2}{\PYZdq{}}\PY{l+s+s2}{ (Coefficients of order 1e\PYZhy{}15 or less may generally be ignored, as these are at roundoff error.)}\PY{l+s+se}{\PYZbs{}n}\PY{l+s+s2}{\PYZdq{}}\PY{p}{)}
    \PY{n}{i} \PY{o}{=} \PY{n}{i}\PY{o}{+}\PY{l+m+mi}{1}
\end{Verbatim}
\end{tcolorbox}

    \begin{Verbatim}[commandchars=\\\{\}]
AB method order: "1".
 When solving y'(t) = y, y(0)=1,
 the first nonzero term should have local truncation error proportional to
O(dt\^{}2) or a higher power of dt.
Subtracting the numerical result from the exact Taylor expansion, we find a
local truncation error of:
  2
dt     ⎛  3⎞
─── + O⎝dt ⎠
 2
 (Coefficients of order 1e-15 or less may generally be ignored, as these are at
roundoff error.)

AB method order: "2".
 When solving y'(t) = y, y(0)=1,
 the first nonzero term should have local truncation error proportional to
O(dt\^{}3) or a higher power of dt.
Subtracting the numerical result from the exact Taylor expansion, we find a
local truncation error of:
    3
5⋅dt     ⎛  4⎞
───── + O⎝dt ⎠
  12
 (Coefficients of order 1e-15 or less may generally be ignored, as these are at
roundoff error.)

AB method order: "3".
 When solving y'(t) = y, y(0)=1,
 the first nonzero term should have local truncation error proportional to
O(dt\^{}4) or a higher power of dt.
Subtracting the numerical result from the exact Taylor expansion, we find a
local truncation error of:
    4
3⋅dt     ⎛  5⎞
───── + O⎝dt ⎠
  8
 (Coefficients of order 1e-15 or less may generally be ignored, as these are at
roundoff error.)

AB method order: "4".
 When solving y'(t) = y, y(0)=1,
 the first nonzero term should have local truncation error proportional to
O(dt\^{}5) or a higher power of dt.
Subtracting the numerical result from the exact Taylor expansion, we find a
local truncation error of:
      5
251⋅dt     ⎛  6⎞
─────── + O⎝dt ⎠
  720
 (Coefficients of order 1e-15 or less may generally be ignored, as these are at
roundoff error.)

AB method order: "5".
 When solving y'(t) = y, y(0)=1,
 the first nonzero term should have local truncation error proportional to
O(dt\^{}6) or a higher power of dt.
Subtracting the numerical result from the exact Taylor expansion, we find a
local truncation error of:
     6
95⋅dt     ⎛  7⎞
────── + O⎝dt ⎠
 288
 (Coefficients of order 1e-15 or less may generally be ignored, as these are at
roundoff error.)

AB method order: "6".
 When solving y'(t) = y, y(0)=1,
 the first nonzero term should have local truncation error proportional to
O(dt\^{}7) or a higher power of dt.
Subtracting the numerical result from the exact Taylor expansion, we find a
local truncation error of:
        7
19087⋅dt     ⎛  8⎞
───────── + O⎝dt ⎠
  60480
 (Coefficients of order 1e-15 or less may generally be ignored, as these are at
roundoff error.)

AB method order: "7".
 When solving y'(t) = y, y(0)=1,
 the first nonzero term should have local truncation error proportional to
O(dt\^{}8) or a higher power of dt.
Subtracting the numerical result from the exact Taylor expansion, we find a
local truncation error of:
       8
5257⋅dt     ⎛  9⎞
──────── + O⎝dt ⎠
 17280
 (Coefficients of order 1e-15 or less may generally be ignored, as these are at
roundoff error.)

AB method order: "8".
 When solving y'(t) = y, y(0)=1,
 the first nonzero term should have local truncation error proportional to
O(dt\^{}9) or a higher power of dt.
Subtracting the numerical result from the exact Taylor expansion, we find a
local truncation error of:
          9
1070017⋅dt     ⎛  10⎞
─────────── + O⎝dt  ⎠
  3628800
 (Coefficients of order 1e-15 or less may generally be ignored, as these are at
roundoff error.)

AB method order: "9".
 When solving y'(t) = y, y(0)=1,
 the first nonzero term should have local truncation error proportional to
O(dt\^{}10) or a higher power of dt.
Subtracting the numerical result from the exact Taylor expansion, we find a
local truncation error of:
        10
25713⋅dt      ⎛  11⎞
────────── + O⎝dt  ⎠
  89600
 (Coefficients of order 1e-15 or less may generally be ignored, as these are at
roundoff error.)

AB method order: "10".
 When solving y'(t) = y, y(0)=1,
 the first nonzero term should have local truncation error proportional to
O(dt\^{}11) or a higher power of dt.
Subtracting the numerical result from the exact Taylor expansion, we find a
local truncation error of:
           11
26842253⋅dt      ⎛  12⎞
───────────── + O⎝dt  ⎠
   95800320
 (Coefficients of order 1e-15 or less may generally be ignored, as these are at
roundoff error.)

AB method order: "11".
 When solving y'(t) = y, y(0)=1,
 the first nonzero term should have local truncation error proportional to
O(dt\^{}12) or a higher power of dt.
Subtracting the numerical result from the exact Taylor expansion, we find a
local truncation error of:
          12
4777223⋅dt      ⎛  13⎞
──────────── + O⎝dt  ⎠
  17418240
 (Coefficients of order 1e-15 or less may generally be ignored, as these are at
roundoff error.)

AB method order: "12".
 When solving y'(t) = y, y(0)=1,
 the first nonzero term should have local truncation error proportional to
O(dt\^{}13) or a higher power of dt.
Subtracting the numerical result from the exact Taylor expansion, we find a
local truncation error of:
               13
703604254357⋅dt      ⎛  14⎞
───────────────── + O⎝dt  ⎠
  2615348736000
 (Coefficients of order 1e-15 or less may generally be ignored, as these are at
roundoff error.)

AB method order: "13".
 When solving y'(t) = y, y(0)=1,
 the first nonzero term should have local truncation error proportional to
O(dt\^{}14) or a higher power of dt.
Subtracting the numerical result from the exact Taylor expansion, we find a
local truncation error of:
               14
106364763817⋅dt      ⎛  15⎞
───────────────── + O⎝dt  ⎠
   402361344000
 (Coefficients of order 1e-15 or less may generally be ignored, as these are at
roundoff error.)

AB method order: "14".
 When solving y'(t) = y, y(0)=1,
 the first nonzero term should have local truncation error proportional to
O(dt\^{}15) or a higher power of dt.
Subtracting the numerical result from the exact Taylor expansion, we find a
local truncation error of:
                15
1166309819657⋅dt      ⎛  16⎞
────────────────── + O⎝dt  ⎠
  4483454976000
 (Coefficients of order 1e-15 or less may generally be ignored, as these are at
roundoff error.)

AB method order: "15".
 When solving y'(t) = y, y(0)=1,
 the first nonzero term should have local truncation error proportional to
O(dt\^{}16) or a higher power of dt.
Subtracting the numerical result from the exact Taylor expansion, we find a
local truncation error of:
           16
25221445⋅dt      ⎛  17⎞
───────────── + O⎝dt  ⎠
   98402304
 (Coefficients of order 1e-15 or less may generally be ignored, as these are at
roundoff error.)

AB method order: "16".
 When solving y'(t) = y, y(0)=1,
 the first nonzero term should have local truncation error proportional to
O(dt\^{}17) or a higher power of dt.
Subtracting the numerical result from the exact Taylor expansion, we find a
local truncation error of:
                   17
8092989203533249⋅dt      ⎛  18⎞
───────────────────── + O⎝dt  ⎠
  32011868528640000
 (Coefficients of order 1e-15 or less may generally be ignored, as these are at
roundoff error.)

AB method order: "17".
 When solving y'(t) = y, y(0)=1,
 the first nonzero term should have local truncation error proportional to
O(dt\^{}18) or a higher power of dt.
Subtracting the numerical result from the exact Taylor expansion, we find a
local truncation error of:
                 18
85455477715379⋅dt      ⎛  19⎞
─────────────────── + O⎝dt  ⎠
  342372925440000
 (Coefficients of order 1e-15 or less may generally be ignored, as these are at
roundoff error.)

AB method order: "18".
 When solving y'(t) = y, y(0)=1,
 the first nonzero term should have local truncation error proportional to
O(dt\^{}19) or a higher power of dt.
Subtracting the numerical result from the exact Taylor expansion, we find a
local truncation error of:
                       19
12600467236042756559⋅dt      ⎛  20⎞
───────────────────────── + O⎝dt  ⎠
   51090942171709440000
 (Coefficients of order 1e-15 or less may generally be ignored, as these are at
roundoff error.)

AB method order: "19".
 When solving y'(t) = y, y(0)=1,
 the first nonzero term should have local truncation error proportional to
O(dt\^{}20) or a higher power of dt.
Subtracting the numerical result from the exact Taylor expansion, we find a
local truncation error of:
                      20
1311546499957236437⋅dt      ⎛  21⎞
──────────────────────── + O⎝dt  ⎠
  5377993912811520000
 (Coefficients of order 1e-15 or less may generally be ignored, as these are at
roundoff error.)

    \end{Verbatim}

    \hypertarget{step-3-output-this-notebook-to-latex-formatted-pdf-file-back-to-top}{%
\section{\texorpdfstring{Step 3: Output this notebook to
\(\LaTeX\)-formatted PDF file {[}Back to
\hyperref[toc]{top}{]}}{Step 3: Output this notebook to \textbackslash LaTeX-formatted PDF file {[}Back to {]}}}\label{step-3-output-this-notebook-to-latex-formatted-pdf-file-back-to-top}}

\[\label{latex_pdf_output}\]

The following code cell converts this Jupyter notebook into a proper,
clickable \(\LaTeX\)-formatted PDF file. After the cell is successfully
run, the generated PDF may be found in the root NRPy+ tutorial
directory, with filename \url{Tutorial-RK_Butcher_Table_Validation.pdf}.
(Note that clicking on this link may not work; you may need to open the
PDF file through another means.)

    \begin{tcolorbox}[breakable, size=fbox, boxrule=1pt, pad at break*=1mm,colback=cellbackground, colframe=cellborder]
\prompt{In}{incolor}{9}{\boxspacing}
\begin{Verbatim}[commandchars=\\\{\}]
\PY{k+kn}{import} \PY{n+nn}{cmdline\PYZus{}helper} \PY{k}{as} \PY{n+nn}{cmd}    \PY{c+c1}{\PYZsh{} NRPy+: Multi\PYZhy{}platform Python command\PYZhy{}line interface}
\PY{n}{cmd}\PY{o}{.}\PY{n}{output\PYZus{}Jupyter\PYZus{}notebook\PYZus{}to\PYZus{}LaTeXed\PYZus{}PDF}\PY{p}{(}\PY{l+s+s2}{\PYZdq{}}\PY{l+s+s2}{Tutorial\PYZhy{}RK\PYZus{}Butcher\PYZus{}Table\PYZus{}Validation}\PY{l+s+s2}{\PYZdq{}}\PY{p}{)}
\end{Verbatim}
\end{tcolorbox}

    \begin{Verbatim}[commandchars=\\\{\}]
Created Tutorial-RK\_Butcher\_Table\_Validation.tex, and compiled LaTeX file
    to PDF file Tutorial-RK\_Butcher\_Table\_Validation.pdf
    \end{Verbatim}


    % Add a bibliography block to the postdoc
    
    
    
\end{document}
